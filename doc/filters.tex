\chapter{Filtering and Sorting of Matches}\indexmain{filters}\label{sub:filters}

When a test or rule was matched, a match or an array of matches is available, with a mapping of the pattern elements to the bound host graph elements. 

Filters may then be used to process the matches array of a rule all application (including the one-element matches array of a single rule application) -- after all matches were found, but before they are rewritten.
They allow you to follow only the \emph{most promising matches} during a search task.

You may implement filters in the form of \emph{filter functions} on your own.
Alternatively, you may declare certain filters at their rules, they are then \emph{auto-generated} for you.
Some filters are \emph{auto-supplied} and may just be used.
All filters are used by \emph{filter calls} from the sequences (normally together with a rule all application).


%%%%%%%%%%%%%%%%%%%%%%%%%%%%%%%%%%%%%%%%%%%%%%%%%%%%%%%%%%%%%%%%%%%%%%%%%%%%%%%%%%%%%%%%%%%%%%%%
\section{Match types}\index{match type}

The match of the pattern of a rule named \texttt{r} has the type \texttt{match<r>}.
The match of the pattern of a first-level iterated \texttt{it} of a rule \texttt{r} has the type \texttt{match<r.it>}.
A match class \texttt{mc} also defines a match type \texttt{match<class mc>}, see Section~\ref{sec:matchclass} for match classes.

\begin{rail}
  MemberAccess: Match '.' Ident;
\end{rail}\ixnterm{MemberAccess}
You can access the members of a match type value in form of the known \emph{MemberAccess} (in the rule language as well as in the sequences), e.g. \texttt{m.n} returns from match \texttt{m} the value bound to pattern element/match class member \texttt{n}.

In Example~\ref{examplebasicpatternmatching}, the match type \texttt{match<r>} contains the nodes \texttt{n:N} and \texttt{m:Node}. The edge \texttt{\_edge0:Edge} only appears at API level, it stems from automatic naming of the anonymous edge in the pattern -- if you want to access a pattern element, assign it a name.

Furthermore, you may execute a \indexed{\texttt{copy}} function to get a clone of a match; it is defined with the signature \texttt{copy(match<T>):match<T>}.\indexmain{copy}
Exceeding these already introduced match type operations, you can also apply a \emph{MatchClassConstructor} \texttt{match<class mc>()} in order to create a value of a match class type; the values of the action match types in contrast are always created by pattern matching.

A rule query with test \texttt{t} specified by \texttt{[?t]} (cf. \ref{sec:patternbasedgraphquery}) yields an  \texttt{array<match<t>>}.
An iterated query specified by \texttt{[?it]} (cf. \ref{sec:primexpr}) yields an \texttt{array<match<t.it>>}.
A multi rule query with test \texttt{t} implementing match class \texttt{mc} specified by \verb#[?[t]\<class mc>]# (also cf. \ref{sec:patternbasedgraphquery}) yields an \texttt{array<match<class mc>>}.
These arrays are to be processed by the array methods introduced in Section~\ref{sec:arrayexpr}.

The other way to process arrays of matches is with filters. In a filter function, the matches array created by pattern matching is made available via the \texttt{this} variable, see below.

%%%%%%%%%%%%%%%%%%%%%%%%%%%%%%%%%%%%%%%%%%%%%%%%%%%%%%%%%%%%%%%%%%%%%%%%%%%%%%%%%%%%%%%%%%%%%%%%
\section{Filter Functions}

Filter functions need to be \emph{defined}, i.e. declared and implemented before they can be \emph{used}.

\begin{rail}
	FilterFunctionDefinition: 'filter' FilterIdentDecl '<' ('class' MatchClassIdent | RuleIdent) '>' \\ (Parameters | ()) lbrace (Statement+) rbrace;
\end{rail}\ixnterm{FilterFunctionDefinition}\ixkeyw{filter}

A \indexed{filter function definition} is given in between the rules as a global construct.
It specifies the name of the filter and the parameters supported, furthermore it specifies which rule (cf. \ref{chaprulelang}) or match class (cf. \ref{cha:multiruleseq}) the filter applies to, and it supplies a list of statements in the body.

The restrictions of a function body apply here, i.e. you are not allowed to manipulate the graph.
In contrast to the function body is a \texttt{this} variable predefined, which gives access to the matches array to filter, of type \texttt{array<match<r>>}, where \texttt{r} denotes the name of the rule the filter is defined for, or of type \texttt{array<match<class mc>>}, where \texttt{mc} denotes the name of the match class the filter is defined for.
All the operations known for variables of array type (cf. \ref{sec:arrayexpr}) are available.

\begin{example}
The following filter \texttt{ff} for the rule \texttt{foo} clones the last match in the array of matches, and adds it to the array.
So the last match would be rewritten twice (which is fine for \texttt{foo} that only creates a node, but take care when deleting or retyping nodes).
But in the following the first entry and the last entry are deleted (again); the first in an efficient in-place way by assigning \texttt{null}.
\begin{grgen}
rule foo
{
	n:Node;
	modify {
		nn:Node;
	}
}

filter ff<foo>
{
	this.add(copy(this.peek())); // note the copy on match<foo>
	this[0] = null; // removes first entry in matches array, efficiently
	this.rem();
}
\end{grgen}
\end{example}

\begin{example}
The following rule \texttt{incidency} yields for each node \texttt{n} matched the number of incident edges to a def variable \texttt{i} and assigns it in the eval on to an attribute \texttt{j}.
The filter \texttt{filterMultiply} modifies the def variable \texttt{i} in the matches with the factor \texttt{f} handed in as filter parameter.
\begin{grgen}
rule incidency
{
	n:N;
---
	def var i:int;
	yield { yield i = incident(n).size(); }
	
	modify {
		eval { n.j = i; }
	}
}
filter filterMultiply<incidency>(var f:int)
{
	for(m:match<incidency> in this)
	{
		m.i = m.i * f;
	}
}
\end{grgen}
The following call triples the count of incident edges that is finally written to the attribute \texttt{j} for each node \texttt{n} matched by \texttt{incidency}:
\begin{grshell}
  exec [incidency \ filterMultiply(3)]
\end{grshell}
\end{example}

You may declare external filters that you have to implement then in a C\# accompanying file, see \ref{sub:extflt} for more on that.
With filter functions and external filter functions, you may implement matches-list modifications exceeding the available pre-implemented functionality.


%%%%%%%%%%%%%%%%%%%%%%%%%%%%%%%%%%%%%%%%%%%%%%%%%%%%%%%%%%%%%%%%%%%%%%%%%%%%%%%%%%%%%%%%%%%%%%%%
\section{Auto-Generated Filters}\label{filter:auto-generated}

Auto-generated filters need to be \emph{declared} at their rule or iterated pattern or match class before they can be \emph{used}, they are implemented by \GrG.

\begin{rail}
  FiltersDecl: ( backslash (FilterDecl + ',') )?;
  FilterDecl: ( FilterIdent '<' (VariableIdent + ',') '>' | 'auto' );
\end{rail}\ixnterm{FiltersDecl}\ixnterm{FilterDecl}\ixkeyw{auto}

The \indexed{auto-generated filters} must be declared at the end of a rule definition (cf. \ref{ruledecls}), or iterated pattern definition (cf. \ref{cardinality}), or match class definition (cf. \ref{sec:matchclass}).
With exception of the \texttt{auto} filter for removal of automorphic matches they have to specify the name of a \texttt{def} or non-\texttt{def} entity contained in the pattern (or in the iterated pattern or in the match class), which may be a variable of one of the basic types or a graph element type (not a container type), or a node or an edge.
They filter based on certain conditions regarding that entity; filtering may mean to reorder a matches list alongside the chosen (def) variable.

The allowed types depend on the filter (and the accumulation method).
The ordering filters utilize the ordering relation, so they support only (def) variables of numeric type (\texttt{byte}, \texttt{short}, \texttt{int}, \texttt{long}, \texttt{float}, \texttt{double},) or \texttt{string} or \texttt{enum} type (also \texttt{boolean} is supported, even if not of much use).
The other filters utilize the equality comparisons, so they support (def) variables of the types offering the ordering relation, plus node and edge types, or nodes or edges (i.e. entities).
The accumulation variable of the \texttt{keepOneForEachAccumulateBy} filter must be of a numeric type, the exact type depends on the result type of the used array accumulation method (\texttt{double} is supported by all of them).

\begin{description}
\item[\texttt{orderAscendingBy<v>}] orders the matches list ascendingly (from lowest to highest value according to the \verb#<# operator), alongside the \texttt{v} contained in each match.
\item[\texttt{orderDescendingBy<v>}] orders the matches list descendingly, alongside the \texttt{v} contained in each match.
\item[\texttt{keepOneForEach<v>}] filters away all matches with duplicate \texttt{v} values, i.e. only one (prototypical) match is kept per \texttt{v} value.
\item[\texttt{keepOneForEach<v>Accumulate<w>By<m>}] filters away all matches with duplicate \texttt{v} values, i.e. only one (prototypical) match is kept per \texttt{v} value. The variable \texttt{w} receives the result of accumulating all \texttt{w} of equal \texttt{v} with the array accumulation method \texttt{m} (see Section \ref{sec:arrayexpr} for the available array accumulation methods).
\item[\texttt{groupBy<v>}] ensures matches of equal \texttt{v} values are neighbours. Grouping can be applied on types that only support the \verb#==# operator.
\item[\texttt{keepSameAsFirst<v>}] filters away all matches with \texttt{v} values that are not equal to the \texttt{v} value of the first match. May be used to ensure all matches of an indeterministically chosen value are rewritten, but typically you want to order the matches list before.
\item[\texttt{keepSameAsLast<v>}] filters away all matches with \texttt{v} values that are not equal to the \texttt{v} value of the last match.
\end{description}

The \texttt{orderAscendingBy<.(,.)*>} and \texttt{orderDescendingBy<.(,.)*>} allow to specify more than one variable, all others (with exception of \texttt{keepOneForEachAccumulateBy} and \texttt{auto}) support only one variable.
The order-by filters order the matches in order of the (def) variables, e.g. \texttt{orderAscendingBy<v,w>} orders the matches primarily by \texttt{v} and secondarily by \texttt{w}, i.e in case the values of the first def variable are equal, the values of the second one decide.
 
Besides those variable-based filters you may use the automorphic matches filter to purge \indexed{symmetric matches}.
In order to do so, specify the special filter named \texttt{auto} at the rule declaration, and use it with the same name at an application of that action (with the known backslash syntax).
The filter is removing matches due to an \indexed{automorphic} pattern, matches which are covering the same spot in the host graph with a permutation of the nodes, the edges, or subpatterns of the same type at the same level.
Other nested pattern constructs which are structurally identical are not recognized to be identical when they appear as commuted subparts; but they are so when they are factored into a subpattern which is instantiated multiple times.
It is highly recommended to use this \indexed{symmetry reduction} technique when building state spaces including isomorphic state collapsing, as puring the matches which lead to isomorphic states early saves expensive graph comparisons -- and often it gives the semantics you are interested in anyway.


%%%%%%%%%%%%%%%%%%%%%%%%%%%%%%%%%%%%%%%%%%%%%%%%%%%%%%%%%%%%%%%%%%%%%%%%%%%%%%%%%%%%%%%%%%%%%%%%
\section{Auto-Supplied Filters}\label{filter:auto-supplied}

A few filters (that do not need information from a specific match) are auto-supplied, they can be used without implementation and even without declaration. They allow to keep or remove a certain number of matches from the beginning or the end of the matches list.

The following auto-supplied filters may be used directly:
\begin{description}
\item[\texttt{keepFirst(count)}] keeps the first \texttt{count} matches from the begin of the matches list, \texttt{count} must be an integer number.
\item[\texttt{keepLast(count) }] keeps the first \texttt{count} matches from the end of the matches list, \texttt{count} must be an integer number.
\item[\texttt{keepFirstFraction(fraction)}] keeps the \texttt{fraction} of the matches from the begin of the matches list, \texttt{fraction} must be a floating point number in between \verb#0.0# and \verb#1.0#.
\item[\texttt{keepLastFraction(fraction)}] keeps the \texttt{fraction} of the matches from the end of the matches list, \texttt{fraction} must be a floating point number in between \verb#0.0# and \verb#1.0#.
\item[\texttt{removeFirst(count)}] removes the first \texttt{count} matches from the begin of the matches list, \texttt{count} must be an integer number.
\item[\texttt{removeLast(count) }] removes the first \texttt{count} matches from the end of the matches list, \texttt{count} must be an integer number.
\item[\texttt{removeFirstFraction(fraction)}] removes the \texttt{fraction} of the matches from the begin of the matches list, \texttt{fraction} must be a floating point number in between \verb#0.0# and \verb#1.0#.
\item[\texttt{removeLastFraction(fraction)}] removes the \texttt{fraction} of the matches from the end of the matches list, \texttt{fraction} must be a floating point number in between \verb#0.0# and \verb#1.0#.
\end{description}

%%%%%%%%%%%%%%%%%%%%%%%%%%%%%%%%%%%%%%%%%%%%%%%%%%%%%%%%%%%%%%%%%%%%%%%%%%%%%%%%%%%%%%%%%%%%%%%%
\section{Filter Calls}\label{sec:filtercalls}

\begin{rail}
	FilterCalls: (backslash FilterCall)*;
	FilterCall: ( FilterIdent ('(' Arguments ')'|()) | FilterIdent '<' VariableIdent '>' | 'auto' );
\end{rail}\ixnterm{FilterCalls}\ixnterm{FilterCall}\ixkeyw{auto}

\begin{rail}
	MatchClassFilterCalls: (backslash MatchClassFilterCall)*;
	MatchClassFilterCall: ( MatchClass '.' FilterIdent ('(' Arguments ')'|()) | MatchClass '.' FilterIdent '<' VariableIdent '>' );
\end{rail}\ixnterm{MatchClassFilterCalls}\ixnterm{MatchClassFilterCall}

Filters are employed with \verb#r\f# notation at rule calls from the sequences language.
This holds for user-implemented, auto-generated, and auto-supplied filters.
Match class filters (defined on the match class) are employed with \verb#[[r]\mc.f]# notation at multi rule calls from the sequences language.
This holds for user-implemented, auto-generated, and auto-supplied filters, too.

%\pagebreak

The normal name resolving rules (regarding packages) apply in an extended version (point 2 added):
\begin{enumerate}
	\item content from the package specified with the prefix, if that is "global" then global content
	\item content from the package the rule or match class is contained in to which the filter applies (so the filter is a dependent entity)
	\item content from package context (which thus overrides global content)
	\item global content 
\end{enumerate}

\begin{example}
The following rule \texttt{deleteNode} yields to a def variable the amount of outgoing edges of a node, and deletes the node.
Without filtering this behaviour would be pointless, but a filter ordering based on the def variable is already declared for it.
\begin{grgen}
rule deleteNode
{
	n:Node;
---
	def var i:int;
	yield {
		yield i = outgoing(n).size();
	}
	
	modify {
		delete(n);
	}
} \ orderAscendingBy<i>
\end{grgen}
The rule may then be applied with a sequence like the following:
\begin{grshell}
  exec [deleteNode \ orderAscendingBy<i> \ keepFirstFraction(0.5)]
\end{grshell}
This way, the 50\% of the nodes with the smallest number of outgoing edges are deleted from the host graph (because they don't get filtered away), or rephrased: the 50\% of the nodes with the highest number of outgoing edges are kept (their matches at the end of the ordered matches list are filtered away).
\end{example}

\pagebreak

%%%%%%%%%%%%%%%%%%%%%%%%%%%%%%%%%%%%%%%%%%%%%%%%%%%%%%%%%%%%%%%%%%%%%%%%%%%%%%%%%%%%%%%%%%%%%%%%
\section{Quick Reference Table}
	
%\makeatletter
\begin{table}[htbp]
\centering
\begin{tabular}{|l|}
\hline
\texttt{orderAscendingBy<var (, var)*>}\\
\texttt{orderDescendingBy<var (, var)*>}\\
\texttt{groupBy<entity>}\\
\texttt{keepSameAsFirst<entity>}\\
\texttt{keepSameAsLast<entity>}\\
\texttt{keepOneForEach<entity>}\\
\texttt{keepOneForEach<entity>Accumulate<accvar>By<method>}\\
\texttt{auto}\\
\hline
\texttt{keepFirst(int)}\\
\texttt{keepLast(int)}\\
\texttt{keepFirstFraction(double)}\\
\texttt{keepLastFraction(double)}\\
\texttt{removeFirst(int)}\\
\texttt{removeLast(int)}\\
\texttt{removeFirstFraction(double)}\\
\texttt{removeLastFraction(double)}\\
\hline
\end{tabular}
\\
{\small Let \texttt{var} be a (def-)variable of a type $\in \{\texttt{byte}, \texttt{short}, \texttt{int}, \texttt{long}, \texttt{float}, \texttt{double}, \texttt{boolean}, \texttt{string}, \texttt{enum}\}$. Let \texttt{entity} be a (def-)variable of one of the types admissible for \texttt{var} or \texttt{Node} or \texttt{Edge} type or a (def-)node or a (def-)edge. Let \texttt{accvar} be a def-variable of a numeric type, the exactly required type depends on the result type of \texttt{method} }}
\caption{Auto-generated and then auto-supplied filters at a glance}
\label{filterstab}
\end{table}

%%%%%%%%%%%%%%%%%%%%%%%%%%%%%%%%%%%%%%%%%%%%%%%%%%%%%%%%%%%%%%%%%%%%%%%%%%%%%%%%%%%%%%%%%%%%%%%%
%\section{Symmetry reduction}
% TODO
%Problem.
%Independent.
%Auto. Define here, above in auto-generated placeholder and reference.

