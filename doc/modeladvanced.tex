\chapter{Object-Oriented and Graph-Oriented Programming}\indexmain{methods}\indexmain{graph-oriented programming}\indexmain{object-oriented programming}
\label{chapmodeladvanced}

In addition to the key features of \GrG\ \emph{graph models}\indexmain{graph model} as already described in Chapter~\ref{chapmodellang}, you may specify \emph{methods} in the node or edge classes.
Methods are functions or procedures that are declared inside class scope.
In contrast to attributes that do not support overwriting -- each attribute of a type is added to all subtypes, with multiple declarations being forbidden -- methods support overwriting.
Independent of the statically known type of the variable, the method that was defined nearest to the exact dynamic type of the value is executed if a method is called.

This \emph{dynamic dispatch} is the defining feature of object-oriented programming.
For graph-oriented programming, the defining feature is \emph{pattern-matching}.
Graph-oriented programming can be made scalable to large tasks with hierarchically \emph{nested graphs}, allowed for by attributes of graph type in the model, implemented with switch-to-subgraph and return-from-subgraph operations.

\begin{rail}
  AdvancedModelDeclarations: () + (FunctionMethodDefinition
										 | ProcedureMethodDefinition
  									 | ExternalClassDeclaration
  									 | ExternalFunctionDeclaration
  									 | EmitParseDeclaration
  									 | CopyCompareDeclaration);
\end{rail}\ixnterm{AdvancedModelDeclarations}

The \emph{ExternalClassDeclaration} registers an \indexed{external class}, including its subtype hierarchy, excluding any attributes with \GrG~which can subsequently be used in attribute computations with \indexed{external function} calls; the \emph{EmitParseDeclaration} declares that the user defines emit and parse functions for external or object type values, whereas the \emph{CopyCompareDeclaration} declares that the user defines copy and compare functions for external or object type values.
You find more on them in \ref{chapextensions}, here we will take a closer look on the \emph{FunctionMethodDefinition} and \emph{ProcedureMethodDefinition}.
%Node/Edge typed attributes are possible.

%%%%%%%%%%%%%%%%%%%%%%%%%%%%%%%%%%%%%%%%%%%%%%%%%%%%%%%%%%%%%%%%%%%%%%%%%%%%%%%%%%%%%%%%%%%%%%%%
\section{Methods}\label{sec:objectoriented}

Computations on attributes of node or edge types that are occurring frequently may be factored out into a method definition given inside a class definition of the model file.
Such compound computations can be built and abstracted into reusable entities in two different forms, \emph{function methods} usable(/callable) from expression context, and \emph{procedure methods} usable(/callable) from statement context.
Besides, some built-in function methods and procedure methods on container types are available in rule language computations and sequence computations, cf. \ref{cha:container};
in addition to the methods defined in \ref{cha:typeexpr} that are only available in the rule language.
%(Furthermore, external functions methods and computation methods may be declared, callable with the same syntax, too.)

%-----------------------------------------------------------------------------------------------
\subsection{Function Method Definition and Call}\label{sub:functionmethods}\label{sec:funcmethcall} 

Function methods are defined in exactly the same way as a function is defined, just inside the hosting class.
They have the same requrirements, 
i.e. exactly one output value must be returned, 
and they must be side-effect free, which especially means that they are not allowed to change the attributes of their hosting type.

\begin{rail} 
  FunctionMethodDefinition: FunctionDefinition;
\end{rail}\ixnterm{FunctionMethodDefinition}

\begin{rail}
  FunctionMethodCall: (Variable | Variable '.' Attribute) '.' Name '(' (Expression * ',') ')' ;
\end{rail}\ixnterm{FunctionMethodCall}

A such defined function method may then be called as an expression atom from anywhere in the rule language file where an expression is required; or even from the sequence computations where an expression is required.
The built-in function methods listed in \ref{funcmethstab} are called with the same syntax.

Inside the function methods, the special entity \texttt{this} is available.
It allows to access the attributes and methods of the type the method is contained in.
In contrast to Java, C++, or C\# where \texttt{this} may be used optionally to denote member or method access,
the usage of \texttt{this} is mandatory in \GrG~to access the attributes of the type or call the methods of the type.

\begin{table}[htbp]
\centering
\begin{tabular}{|l|}
\hline
\texttt{string.length():int}\\
\texttt{string.indexOf(string):int}\\
\texttt{string.lastIndexOf(string):int}\\
\texttt{string.substring(int, int):string}\\
\texttt{string.replace(int, int, string):string}\\
\hline
\texttt{set<T>.size():int}\\
\texttt{set<T>.empty():boolean}\\
\texttt{set<T>.peek(int):T}\\
\hline
\texttt{map<S,T>.size():int}\\
\texttt{map<S,T>.empty():boolean}\\
\texttt{map<S,T>.peek(int):S}\\
\texttt{map<S,T>.domain():set<S>}\\
\texttt{map<S,T>.range():set<T>}\\
\hline
\texttt{array<T>.size():int}\\
\texttt{array<T>.empty():boolean}\\
\texttt{array<T>.peek([int]):T}\\
\texttt{array<T>.indexOf(T):int}\\
\texttt{array<T>.lastIndexOf(T):int}\\
\texttt{array<T>.subarray(int, int):array<T>}\\
\hline
\texttt{deque<T>.size():int}\\
\texttt{deque<T>.empty():boolean}\\
\texttt{deque<T>.peek(num):T}\\
\texttt{deque<T>.indexOf(T):int}\\
\texttt{deque<T>.lastIndexOf(T):int}\\
\texttt{deque<T>.subdeque(int, int):deque<T>}\\
\hline
\end{tabular}
\caption{Function methods at a glance}
\label{funcmethstab}
\end{table}

%-----------------------------------------------------------------------------------------------
\subsection{Procedure Method Definition And Call}\label{sub:proceduremethods}\label{sec:procmethcall} 

Procedure methods are defined in exactly the same way as a procedure is defined, just inside the hosting class.
They may return an arbitrary number of return values, and are thus only callable as a statement.
They are allowed to manipulate the graph and esp. the element they are contained in as needed while being executed;
so you are free to call other procedures -- but in contrast to global procedures, you can not include exec statements.

\begin{rail} 
  ProcedureMethodDefinition: ProcedureDefinition;
\end{rail}\ixnterm{ProcedureMethodDefinition}

\begin{rail}
  ProcedureMethodCall: (OutputAssignment)? (Variable | Variable '.' Attribute) \\
		'.' Name '(' (Expression * ',') ')' ;
\end{rail}\ixnterm{ProcedureMethodCall}

A such defined procedure may then be called as a statement atom from anywhere in the rule language file where an attribute evaluation (/computation) is required; or even from the sequence computations where a statement is required.
The built-in procedure methods listed in \ref{procmethstab} are called with the same syntax.

Inside the procedure method, the special entity \texttt{this} is available.
It allows to access the attributes and methods of the type the method is contained in.
In contrast to Java, C++, or C\# where \texttt{this} may be used optionally to denote member or method access,
the usage of \texttt{this} is mandatory in \GrG~to access the attributes of the type or call the methods of the type.


%\makeatletter
\begin{table}[htbp]
\centering
\begin{tabular}{|l|}
\hline
\texttt{set<T>.add(T)}\\
\texttt{set<T>.rem(T)}\\
\texttt{set<T>.clear()}\\
\hline
\texttt{map<S,T>.add(S,T)}\\
\texttt{map<S,T>.rem(S)}\\
\texttt{map<S,T>.clear()}\\
\hline
\texttt{array<T>.add(T[,int])}\\
\texttt{array<T>.rem([int])}\\
\texttt{array<T>.clear()}\\
\hline
\texttt{deque<T>.add(T[,int])}\\
\texttt{deque<T>.rem([int])}\\
\texttt{deque<T>.clear()}\\
\hline
\end{tabular}
\caption{Procedure methods at a glance}
\label{procmethstab}
\end{table}


\begin{example}
In the following listing, we declare function and a procedure methods inside a node class, and use them from a rule.
Please note the mandatory \texttt{this} in the methods to denote members and call other methods.
	\begin{grgen}
node class N
{
	i:int;
	
	function get_i():int
	{
		return(this.i);
	}
	procedure set_i(var val:int)
	{
		this.i = val;
		return;
	}
	procedure inc_i(var val:int) : (int)
	{
		this.set_i( this.get_i() + val );
		return(this.i);
	}
}

rule foo : (int)
{
	n:N;
	modify {
		def var i:int;
		eval {
			n.set_i(41);
			n.inc_i(1);
			yield i = n.get_i();
		}
		return(i); // exeuting foo will return 42
	}
}
	\end{grgen}
\end{example}


\begin{example}
The following listing highlights the effect of dynamic dispatch (the heart and soul of object-oriented programming): 
the method finally called depends not simply on the static type we know for sure the object is an instance of,
but on the real dynamic type of the object, that it was created with.
Please note that in reality you use this language device to achieve the \emph{same behaviour} as seen from the outward of objects of different types, but implemented \emph{internally consistent} regarding the exact type, taking all of its members into account (including the ones unknown to parent types).
	\begin{grgen}
node class N
{	
	function get():int
	{
		return(0);
	}
}

node class NN extends N
{	
	function get():int
	{
		return(1);
	}
}

rule foo : (int)
{
	n:N; // statically declared N, may match dynamically to node of type NN
	modify {
		def var i:int;
		eval {
			yield i = n.get();
		}
		return(i); // exeuting foo will return 0 iff n was bound to node of type N, but 1 iff n was bound to node of type NN
	}
}
	\end{grgen}
\end{example}



%%%%%%%%%%%%%%%%%%%%%%%%%%%%%%%%%%%%%%%%%%%%%%%%%%%%%%%%%%%%%%%%%%%%%%%%%%%%%%%%%%%%%%%%%%%%%%%%
\section{Graph Nesting}\label{sec:graphnesting}

Graph nesting is possible with nodes or edges of a graph bearing attributes of graph type (cf. \ref{cha:graph}).
They can then be filled with other graphs, employing the operations in \ref{sec:subgraphop}.

Attributes of graph type are opaque to the processing of the host graph containing them; 
their only direct usage is comparison, i.e. isomorphy checking against other graphs,
as needed for state space enumeration (cf. \ref{sec:statespaceenum}),
in this case only immutable subgraphs are stored.

\begin{rail} 
  ExtendedControl: 
    'in' (GraphVariable | NodeOrEdge '.' Attribute) lbrace RewriteSequence rbrace;
\end{rail}\ixkeyw{in}

But they can be opened up and made modifiable by switching the location of processing with the \verb#in g { seq }# sequence.
Inside the braces is the host graph switched to \texttt{g}, the sequence \texttt{seq} is executed in the host graph switched to, all queries and updates are carried out on the new graph; after executing the construct the old graph that was previously used is made the current host graph again.

\begin{rail}
  Rule: GraphVariable '.' RuleIdent (() | '(' (Variable+',') ')');
\end{rail}\ixnterm{Rule}

The rules are always applied on the current host graph without any ability to switch to a subgraph;
switching the location of processing is only available in the sequences.
But a simplified and more lightweight version of the full switch is available for single rule calls with \verb#g.r#; so a method call in the plain sequences (above the sequence computations) denotes in fact a temporary subgraph switch.
The syntax of rule application is extended by the grammar rule above to allow for this is.

The \emph{information hiding} shown by the graph attributes is comparable to the information hiding shown by the objects in \emph{object-oriented programming}, there the attributes but especially the neighbouring elements are only known to the containing object and accessible to the methods of the object.
In \emph{graph-oriented programming} are the attributes but especially the neighboring elements known to the containing graph, the connecting topology is open for \emph{pattern matching}.
This crucial difference also defines the main benefit compared to OO, removing it would mean to revert back to OO.
But this openness might not be needed always for all parts.
When building a \emph{large system}, you typically only need a certain \emph{layer} to be accessible at a time.
You may use graph attributes and nested graph in this case,
utilizing open graph-oriented programming for the parts you need to work globally with pattern matching at a time,
and closed object-oriented programming for parts you only need to access locally,
decoupled by explicit move-to-subgraph and return-to-subgraph steps.

You may model containment with edges denoting a containment type pointing to the contained parts instead of attributes of subgraph type
when the pattern matcher needs overall access to the graph, but there are still some containment or nesting relationships in place.
You can then still benefit from a hierarchical structure in debugging, utilizing the built-in nesting for visualization capabilities of \GrG (cf. \ref{sub:visual}, the graphs nested in attributes are truly opaque and invisible, only when processing switches to them are they displayed instead of their containing host graph).

