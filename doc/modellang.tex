\chapter{Graph Model Language}
The key features of \GrG\ graph models from \cite{geiss}:

\begin{description}
\item[Types.] Nodes and edges can have types (classes). This is similar to common programming languages, except \GrG\ types have no concept of methods. 
\item[Attributes.] Nodes and edges can possess attributes. The set of attributes assigned to a node or edge is determined by its type. The attributes itself are typed, too.
\item[Inheritance.] Types (classes) can be composed by multiple inheritance. \emph{Node} and \emph{Edge} are built-in root types of node and edge types, respectively. Inheritance eases the specification of attributes, because subtypes inherit the attributes of their super types. Note that \GrG\ lacks a concept of overwriting. On a path in the type hierarchy graph from a type up to the built-in root type there must be exactly one declaration for each attribute identifier.
\item[Connection Assertions.] To specify that certain edge types should only connect specific nodes, we include connection assertions. Furthermore the number of outgoing and incoming edges can be constrained.
\end{description}

\section{Building Blocks}

\emph{Note:} The following syntax specifications make heavy use of syntax diagrams (also known as rail diagrams). Syntax diagrams provide an visualization of EBNF grammars. Follow a path along the arrows from left to right through a diagram to get a valid sentence (or sub sentence) of the language. Ellipses are terminals whereas rectangles are non-terminals. For further information on syntax diagrams see \cite{

Basic elements of the \GrG\ graph model language are numbers and identifiers to denominate types, fields and the model itself. The \GrG\ graph model language is case sensitive.\\
\\
\emph{Ident}\\
A character sequence of arbitrary length consisting of letters, digits or underscores. The first character must not be a digit.\\
\\
\emph{NodeType}, \emph{EdgeType}, \emph{EnumType}\\
These are (semantic) specializations of Ident to restrict an identifier to a specific type.\\
\\
\emph{Number}\\
A sequence of digits. The sequence has to form a non-negative integer in decade system and will be internally stored in 32 bit tow's complement representation.

\section{Type Declarations}
\begin{rail}
  GraphModel: 'model' Ident ';' (() + TypeDeclaration);
\end{rail}
The graph model consists of its name \emph{Ident} and type declarations defining the type of nodes and edges.

\begin{rail}
  TypeDeclaration: EnumDeclaration | ClassDeclaration
\end{rail}
The primitive types \emph{int}, \emph{boolean} and \emph{string} are predefined. Types does not need to be declared before they are used.

\begin{rail}
  EnumDeclaration: 'enum' EnumType lbrace ((Ident (() | '=' IntExpr)) + ',') rbrace ;
  ClassDeclaration: (() | 'abstract') (() | 'const') (NodeClass | EdgeClass);
  NodeClass: 'node' 'class' NodeType (() | 'extends' (NodeType+',')) \\ 
    (';' | lbrace FieldDeclarations rbrace);
  EdgeClass: 'edge' 'class' EdgeType (() | 'extends' (EdgeType+',')) \\
    (() + ConnectAssertions) (';' | lbrace FieldDeclarations rbrace);
  FieldDeclarations: (() | IdentDeclaration ':' FieldType ';') + ; 
  IdentDeclaration: Ident (() | '[' (Ident '=' Constant + ',') ']');
  FieldType: PrimitiveType | EnumType ; 
  ConnectAssertions: 'connect' (NodeConstraint '->' NodeConstraint + ',');
  NodeConstraint: NodeType (() | '[' ('*' | '+' | Number | RangeConstraint) ']') ;
  RangeConstraint: Number ':' ('*' | Number) ;
\end{rail}

 