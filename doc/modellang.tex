\chapter{Graph Model Language}\indexmain{graph model language}
\label{chapmodellang}
The key features of \GrG\ \emph{graph models}\indexmain{graph model} as described by Geiß et al.~\cite{GBGHS:06,KG:07} are given below:

\begin{description}
\item[Types] Nodes and edges are typed. 
  This is similar to classes in common programming languages, except for the concept of methods that \GrG\ nodes and edges don't support. 
\item[Attributes] Nodes and edges can possess attributes. The set of attributes assigned to a node or edge is determined by its type. The attributes themselves are typed, too.
\item[Inheritance] Node and edge types (classes) can be composed by multiple \indexed{inheritance}. \texttt{Node} and \texttt{Edge} are built-in root types of node and edge types, respectively. Inheritance eases the specification of attributes because subtypes inherit the attributes of their super types. Note that \GrG\ lacks a concept of overwriting. On a path in the \indexed{type hierarchy} graph from a type up to the built-in root type there must be exactly one declaration for each attribute identifier. Furthermore if multiple paths from a type up to the built-in root type exist, the declaring types for an attribute identifier must be the same on all such paths.
\item[Connection Assertions] To specify that certain edge types should only connect specific nodes, we include connection assertions. Furthermore the number of outgoing and incoming edges can be constrained.
\end{description}

\begin{figure}[htbf]
\begin{example}\label{ex:model:map}
The following toy example of a model of road maps gives a rough picture of the language:
\begin{grgen}
model Map;

enum resident {village = 500, town = 5000, city = 50000}

node class sight;

node class city {
	size: resident;
}

const node class metropolis extends city {
  river: string;
}  

abstract node class abandoned_city extends city;
node class ghost_town extends abandoned_city;

edge class street;
edge class trail extends street;
edge class highway extends street
    connect metropolis [+] -> metropolis [+]
{
    jam: boolean;
}
\end{grgen}
\end{example}
\end{figure}
In this chapter as well as in chapter~\ref{chapgrshell} (\GrShell) we use excerpts of Example~\ref{ex:model:map} (the \texttt{Map} model) for illustration purposes.

\section{Building Blocks}
\label{modelbb}

\begin{note}
The following syntax specifications make heavy use of \newtermsee{syntax diagram}{rail diagram}s (also known as \indexed{rail diagram}s). Syntax diagrams provide a visualization of EBNF\footnote{Extended Backus–Naur Form.} grammars. Follow a path along the arrows through a diagram to get a valid sentence (or subsentence) of the language. Ellipses represent terminals whereas rectangles represent non-terminals. For further information on syntax diagrams see~\cite{MMJW:91}.
\end{note}
Basic elements of the \GrG\ graph model language are identifiers to denominate types, attributes, and the model itself. The \GrG\ graph model language is \indexed{case sensitive}.\\
\\
\emph{Ident}, \emph{IdentDecl}\\ \indexmain{identifier}\nopagebreak
A non-empty character sequence of arbitrary length consisting of letters, digits, or underscores. The first character must not be a digit. \emph{Ident} and \emph{IdentDecl} differ in their role: While \emph{IdentDecl} is a \emph{defining} occurrence of an identifier, \emph{Ident} is a \emph{using} occurrence. An \emph{IdentDecl} non-terminal can be annotated\indexmain{annotation}. See Section~\ref{annotations} for annotations of declarations.
\begin{note}
\label{note:modeldecl}
  The \GrG\ model language does not distinguish between \indexed{declaration}s and \indexed{definition}s. More precisely, every declaration is also a definition. For instance, the following C-like pseudo \GrG\ model language code is illegal:
\begin{grgen}
node class t_node;
node class t_node {
  ...
}
\end{grgen}
Using an identifier before defining it is allowed. Every used identifier has to be defined exactly once.
\end{note}
\pagebreak
\emph{NodeType}, \emph{EdgeType}, \emph{EnumType}\\ \nopagebreak
These are (semantic) specializations of \emph{Ident} to restrict an identifier to denote a node type, an edge type, or an enum type, respectively.

\section{Type Declarations}
\begin{rail}
  GraphModel: 'model' IdentDecl ';' (() + TypeDeclaration);
\end{rail}\ixkeyw{model}\ixnterm{GraphModel}
The \indexed{graph model} consists of its name \emph{IdentDecl} and type declarations defining specific node and edge types as well as enums.

\begin{rail}
  TypeDeclaration: EnumDeclaration | ClassDeclaration
\end{rail}\ixnterm{TypeDeclaration}
\emph{ClassDeclaration} defines a node type or an edge type. \emph{EnumDeclaration} defines an enum type for use as attribute of nodes or edges. Like all identifier definitions, types do not need to be declared\indexmain{declaration} before they are used.

\begin{rail}
  EnumDeclaration: 'enum' IdentDecl lbrace ((IdentDecl (() | '=' IntExpr)) + ',') rbrace ;
\end{rail}\ixkeyw{enum}\ixnterm{EnumDeclaration}
Defines an \indexed{enum type}.
A \GrG\ enum is internally represented as \texttt{int} (see Section~\ref{builtin}).

\begin{example}
\begin{grgen}
enum Color {red, green, blue}
enum Resident {village = 500, town = 5000, city = 50000}
enum AsInC {a = 2, b, c = 1, d, e = (int)Resident::village + c}
\end{grgen}
The semantics is as in C~\cite{Sch:1990:ANSIC}. So, the following holds: $\texttt{red} = 0$, $\texttt{green} = 1$, $\texttt{blue} = 2$, $\texttt{a}=2$, $\texttt{b}=3$, $\texttt{c}=1$, $\texttt{d}=2$, and $\texttt{e}=501$.
\end{example}

\begin{rail}  
  ClassDeclaration: (() | 'abstract') (() | 'const') (NodeClass | EdgeClass);
\end{rail}\ixkeyw{abstract}\ixkeyw{const}\ixnterm{ClassDeclaration}
Defines a new node type or edge type. The keyword \texttt{abstract} indicates that you cannot instantiate graph elements of this type. Instead you have to derive non-abstract types to create graph elements. The abstract-property will not be inherited by subclasses, of course.

\begin{example}
We adjust our map model and make \texttt{city} abstract:
\begin{grgen}
abstract node class city {
	size: int;
}
abstract node class abandoned_city extends city;
node class ghost_town extends abandoned_city;
\end{grgen}
You will be able to create nodes of type \texttt{ghost\_town}, but not of type \texttt{city} or \texttt{abandoned\_city}. However, nodes of type \texttt{ghost\_town} are also of type \texttt{abandoned\_city} as well as of type \texttt{city} and they have the attribute \texttt{size}, hence.
\end{example}
The keyword \texttt{const} indicates that rules may not write to attributes (see also Section~\ref{replacepart}, \texttt{eval}). However, such attributes are still writable by \LibGr\indexmain{libGr} and \GrShell\indexmain{GrShell} directly. This property applies to attributes defined in the current class, only. It does not apply to inherited attributes. The \texttt{const} property will not be inherited by subclasses, either. If you want a subclass to have the \texttt{const} property, you have to set the \texttt{const} modifier explicitly.

\begin{rail}  
  NodeClass: 'node' 'class' IdentDecl (() | 'extends' (NodeType+',')) \\ 
    (';' | lbrace AttributeDeclarations rbrace);
\end{rail}\ixkeyw{node}\ixkeyw{class}\ixkeyw{extends}\ixnterm{NodeClass}
Defines a new \indexed{node type}. Node types can inherit\indexmain{inheritance} from other node types defined within the same file. If the \texttt{extends} clause is omitted, \emph{NodeType} will inherit from the built-in type \texttt{\indexed{Node}}. Optionally nodes can possess attributes.

\begin{rail}    
  EdgeClass: 'edge' 'class' IdentDecl (() | 'extends' (EdgeType+',')) \\
    (() + ConnectAssertions) (';' | lbrace AttributeDeclarations rbrace);
\end{rail}\ixkeyw{edge}\ixkeyw{class}\ixkeyw{extends}\ixnterm{EdgeClass}
Defines a new \indexed{edge type}. Edge types can inherit\indexmain{inheritance} from other edge types defined within the same file. If the \texttt{extends} clause is omitted, \emph{EdgeType} will inherit from the built-in type \texttt{\indexed{Edge}}. Optionally edges can possess attributes. A \newterm{connection assertion} specifies that certain edge types should only connect specific nodes. Moreover, the number of outgoing and incoming edges can be constrained.

\begin{note}
It is not forbidden to create graphs that are invalid according to \indexed{connection assertion}s. \GrG\ just enables you to check, whether a graph is valid or not. See also Section~\ref{graphcommands}, \texttt{validate}.
\end{note}

\begin{rail}  
  ConnectAssertions: 'connect' (NodeConstraint '->' NodeConstraint + ',');
  NodeConstraint: NodeType (() | '[' ('*' | '+' | Number | RangeConstraint) ']') ;
  RangeConstraint: Number ':' ('*' | Number) ;
\end{rail}\ixkeyw{connect}\ixnterm{ConnectAssertions}\ixnterm{NodeConstraint}\ixnterm{RangeConstraint}
A \indexed{connection assertion} is denoted as a pair of node types in conjunction with their multiplicities\indexmainsee{multiplicity}{connection assertion}. A corresponding edge may connect a node of the first node type or one of its subtypes (source) with a node of the second node type or one of its subtypes (target). The multiplicity is a constraint on the out-degree and in-degree\indexmainsee{degree}{connection assertion} of the source and target node type, respectively. \emph{Number} is an \texttt{int} constant as defined in Section~\ref{expressions}. See Section~\ref{graphcommands}, \texttt{validate}\ixkeyw{validate}, for an example. Table~\ref{multiplicities} describes the multiplicity definitions.
\begin{table}[htbp]
\begin{tabularx}{\linewidth}{|l|X|}\hline
	\texttt{[$n$:*]} & The number of edges incident to a node of that type is unbounded. At least $n$ edges must be incident to nodes of that type.\\ 
	\texttt{[$n$:$m$]} & At least $n$ edges must be incident to nodes of that type, but at most $m$ edges may be incident to nodes of that type ($m \geq n$ must hold).\\
	\texttt{[*]} & Abbreviation for \texttt{[0:*]}.\\
	\texttt{[+]} & Abbreviation for \texttt{[1:*]}.\\
	\texttt{[$n$]} & Abbreviation for \texttt{[$n$:$n$]}. \\ \hline
\end{tabularx}
\caption{\GrG\ node constraint multiplicities}
\label{multiplicities}
\end{table}

\begin{rail}    
  AttributeDeclarations: (() | IdentDecl ':' AttributeType ';') + ;
  AttributeType: PrimitiveType | EnumType ; 
\end{rail}\ixnterm{AttributeDeclarations}\ixnterm{AttributeType}
Defines a node or edge \indexed{attribute}. Possible types are enumeration types (\texttt{enum}) and primitive types. See Section~\ref{builtin} for a list of built-in primitive types.



