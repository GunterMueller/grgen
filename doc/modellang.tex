\chapter{Graph Model Language}
\label{chapmodellang}
The key features of \GrG\ graph models from \cite{geiss}:

\begin{description}
\item[Types.] Nodes and edges can have types (classes). This is similar to common programming languages, except \GrG\ types have no concept of methods. 
\item[Attributes.] Nodes and edges can possess attributes. The set of attributes assigned to a node or edge is determined by its type. The attributes themselfs are typed, too.
\item[Inheritance.] Types (classes) can be composed by multiple inheritance. \emph{Node} and \emph{Edge} are built-in root types of node and edge types, respectively. Inheritance eases the specification of attributes, because subtypes inherit the attributes of their super types. Note that \GrG\ lacks a concept of overwriting. On a path in the type hierarchy graph from a type up to the built-in root type there must be exactly one declaration for each attribute identifier.
\item[Connection Assertions.] To specify that certain edge types should only connect specific nodes, we include connection assertions. Furthermore the number of outgoing and incoming edges can be constrained.
\end{description}

\begin{figure}[htbf]
\begin{example}\label{ex:model:map}
The following micro model of street maps gives a rough picture of the language:
\begin{grgen}
model Map;

enum resident {village = 500, town = 5000, city = 50000}

node class sight;

node class city {
	size: resident;
}

const node class metropolis extends city {
  river: string;
}  

abstract node class abandoned_city extends city;
node class ghost_town extends abandoned_city;

edge class street;
edge class trail extends street;
edge class highway extends street
    connect metropolis [+] -> metropolis [+]
{
    jam: boolean;
}
\end{grgen}
\end{example}
\end{figure}
In this chapter as well as in the \GrShell\ chapter \ref{chapgrshell} we use pieces of the example~\ref{ex:model:map} (the \texttt{Map} model) for further descriptions.

\section{Building Blocks}
\label{modelbb}

\begin{note}
The following syntax specifications make heavy use of syntax diagrams (also known as rail diagrams). Syntax diagrams provide a visualization of EBNF grammars. Follow a path along the arrows from left to right through a diagram to get a valid sentence (or sub sentence) of the language. Ellipses are terminals whereas rectangles are non-terminals. For further information on syntax diagrams see \cite{pascal}.
\end{note}
Basic elements of the \GrG\ graph model language are numbers and identifiers to denominate types, fields and the model itself. The \GrG\ graph model language is case sensitive.\\
\\
\emph{Ident}, \emph{IdentDecl}\\ \nopagebreak
A character sequence of arbitrary length consisting of letters, digits or underscores. The first character must not be a digit. \emph{Ident} and \emph{IdentDecl} differ in their role: while \emph{IdentDecl} is a \emph{defining} occurrence of an identifier, \emph{Ident} is a \emph{using} occurence. An \emph{IdentDecl} non-terminal can be annotated. See \ref{annotations} for annotations on declarations.
\begin{note}
  The \GrG\ model language does not distinguish between declarations and definitions. More precisely, every declaration is also a definition.
\end{note}
\mbox{ }\\
\emph{NodeType}, \emph{EdgeType}, \emph{EnumType}\\ \nopagebreak
These are (semantic) specializations of Ident to restrict an identifier to a specific type.\\
\\
\emph{Number}\\ \nopagebreak
A sequence of digits. The sequence has to form a non-negative integer in the decimal system and will be internally stored in 32 bit two's complement representation.

\section{Type Declarations}
\begin{rail}
  GraphModel: 'model' IdentDecl ';' (() + TypeDeclaration);
\end{rail}
The graph model consists of its name \emph{IdentDecl} and type declarations defining specific node and edge types as well as enums.

\begin{rail}
  TypeDeclaration: EnumDeclaration | ClassDeclaration
\end{rail}
\emph{ClassDeclaration} defines a node or an edge. \emph{EnumDeclaration} defines an enum type for use as attribute of nodes or edges. Types do not need to be declared before they are used.

\begin{rail}
  EnumDeclaration: 'enum' IdentDecl lbrace ((IdentDecl (() | '=' IntExpr)) + ',') rbrace ;
\end{rail}
Defines an enum type.

\begin{example}
\begin{grgen}
enum Color {red, green, blue}
enum Resident {village = 500, town = 5000, city = 50000}
enum AsInC {a = 2, b, c = 1, d, e = Resident::village + c}
\end{grgen}
The semantics is as it is in C \cite{isoc}. So the examples hold $\text{red} = 0$, $a=d=2$, $b=3$, $c=1$ and $e=501$.
\end{example}

\begin{rail}  
  ClassDeclaration: (() | 'abstract') (() | 'const') (NodeClass | EdgeClass);
\end{rail}
Defines a new node type or edge type.\\
The keyword \texttt{abstract} indicates that you can't instantiate graph elements of this type but rather you have to derive non-abstract types for graph elements. The abstract-property will not be inherited by subclasses.

\begin{example}
We adjust our map model and make \texttt{city} abstract:
\begin{grgen}
abstract node class city {
	size: int;
}
abstract node class abandoned_city extends city;
node class ghost_town extends abandoned_city;
\end{grgen}
You will be able to create nodes of type \texttt{ghost\_town}, but not of type \texttt{city} or \texttt{abandoned\_city}. However, nodes of type \texttt{ghost\_town} are also nodes of type \texttt{abandoned\_city} and of type \texttt{city} and they have got the attribute \texttt{size}.
\end{example}

The keyword \texttt{const} indicates that rules may not write to attributes. See also \ref{replacepart}, \texttt{eval}. However, attributes are writable by \LibGr\ and \GrShell. This property will not be inherited by subclasses. If you want a subclass to have constant attributes, you have to set the \texttt{const} modifier explicitly.

\begin{rail}  
  NodeClass: 'node' 'class' IdentDecl (() | 'extends' (NodeType+',')) \\ 
    (';' | lbrace FieldDeclarations rbrace);
\end{rail}
Defines a new node type. Node types can inherit from other node types defined within the same file. If the \texttt{extends} clause is omitted, \emph{NodeType} will inherit from the built-in type \texttt{Node}. Optionally nodes can possess attributes (fields).

\begin{rail}    
  EdgeClass: 'edge' 'class' IdentDecl (() | 'extends' (EdgeType+',')) \\
    (() + ConnectAssertions) (';' | lbrace FieldDeclarations rbrace);
\end{rail}
Defines a new edge type. Edge types can inherit from other edge types defined within the same file. If the \texttt{extends} clause is omitted, \emph{EdgeType} will inherit from the built-in type \texttt{Edge}. Optionally edges can possess attributes (fields). A \emph{connection assertion} specifies that certain edge types should only connect specific nodes and, moreover, the number of outgoing and incoming edges can be constrained.

\begin{rail}  
  ConnectAssertions: 'connect' (NodeConstraint '->' NodeConstraint + ',');
  NodeConstraint: NodeType (() | '[' ('*' | '+' | Number | RangeConstraint) ']') ;
  RangeConstraint: Number ':' ('*' | Number) ;
\end{rail}
A connection assertion is described as a pair of node types, optionally together with a multiplicity. A corresponding edge may connect a node of the first node type or one of its subtypes (source) with a node of the second node type or one of its subtypes (destination). The multiplicity is a constraint on the out-degree and in-degree of the source and destination node type respectively. See \ref{graphcommands}, \texttt{validate}, for an example. Table \ref{multiplicities} describes the multiplicity definitions.
\begin{table}[htbp]
\begin{tabularx}{\linewidth}{|l|X|}\hline
	\texttt{[*]} & Number of edges the node is adjacent to is unbounded. The node must not be connected to any edge. This is the \textbf{default}.\\
	\texttt{[+]} & Number of edges the node is adjacent to is unbounded. At least one edge must be adjacent to nodes of that type.\\
	\texttt{[n:*]} & Number of edges the node is adjacent to is unbounded. At least $n$ edges must be adjacent to nodes of that type.\\ 
	\texttt{[n:m]} & At least $n$ edges must be adjacent to the nodes of that type, but at most $m$ edges may be adjacent to the nodes of that type ($m \geq n$ holds).\\
	\texttt{[n]} & Abbreviation for \texttt{[n:n]}. \\ \hline
\end{tabularx}
\caption{\GrG\ node constraint multiplicities concerning a specific pair of an edge type and a node type.}
\label{multiplicities}
\end{table}

\begin{rail}    
  FieldDeclarations: (() | IdentDecl ':' FieldType ';') + ;
  FieldType: PrimitiveType | EnumType ; 
\end{rail}
Defines a node or edge attribute. Possible types are \texttt{enum} and primitive types. See \ref{builtin} for a list of built-in primitive types.

