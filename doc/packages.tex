%\usepackage{ucs} veraltet?
\usepackage[utf8]{inputenc}
%\PrerenderUnicode{Ä}
%\PrerenderUnicode{Ö}
%\PrerenderUnicode{Ü}
%\PrerenderUnicode{ä}
%\PrerenderUnicode{ö}
%\PrerenderUnicode{ü}
%\PrerenderUnicode{ß}

\usepackage{german}
\usepackage[english, german]{babel}
%\usepackage[T1]{fontenc} 
\usepackage[left]{eurosym}
\usepackage{tabularx}
\newcolumntype{v}[1]{%
  >{\raggedright\hspace{Opt}}p{#1}%
}  


%\usepackage{hyperref}

\usepackage{amsmath}
\usepackage{amsfonts}
\usepackage{amssymb}
%\usepackage{bbm}
\usepackage{leftidx}
\usepackage{nicefrac}

%\usepackage{pslatex}
%\usepackage{pdftricks}
%\begin{psinputs}
%   \usepackage{color}
%   \definecolor{light}{gray}{0.8}
%   \usepackage[all, dvips]{xy}
%\end{psinputs} 
%\usepackage{setspace}
%\usepackage{chngpage}
\usepackage{thumbpdf}

\usepackage[all]{xy}
\usepackage{xcolor}
\definecolor{light}{gray}{0.8}
\definecolor{tuerkis}{cmyk}{0.5,0.15,0,0.3}
\usepackage{graphicx}
%\usepackage{pdfpages}
%\usepackage{overpic}
%\usepackage[absolute, overlay]{textpos}
\usepackage{calc}

%\usepackage{listings}
%\lstset{language=C, numbers=left, numberstyle=\tiny, stepnumber=5}

\providecommand{\O}[1]{\ensuremath{\mathcal{O}\left( #1 \right)}}

\providecommand{\N}{\ensuremath{\mathbb{N}}}
\providecommand{\Z}{\ensuremath{\mathbb{Z}}}
\renewcommand{\P}{\ensuremath{\mathbb{P}}}
\providecommand{\F}{\ensuremath{\mathbb{F}}}
\providecommand{\Q}{\ensuremath{\mathbb{Q}}}
\providecommand{\R}{\ensuremath{\mathbb{R}}}
\providecommand{\C}{\ensuremath{\mathbb{C}}}

\providecommand{\abs}[1]{\left\lvert #1 \right\rvert}
\providecommand{\norm}[1]{\left\lVert #1 \right\rVert}
\providecommand{\floor}[1]{\left\lfloor #1 \right\rfloor}
\providecommand{\svert}{\; \vert \;}

\DeclareMathOperator{\grad}{grad} % Gradient
\DeclareMathOperator{\rot}{rot}   % Rotation
\DeclareMathOperator{\Div}{div}   % Divergenz

\DeclareMathOperator{\rg}{rg}     % Rang
\DeclareMathOperator{\Grad}{Grad} % Grad (eines Polynoms)

\DeclareMathOperator{\Abb}{Abb}   % Abbildungsmenge
\DeclareMathOperator{\Sym}{Sym}   % Bijektionenmenge
\DeclareMathOperator{\Hom}{Hom}   % Homomorphismenmenge
\DeclareMathOperator{\End}{End}   % Endomorphismenmenge
\DeclareMathOperator{\Aut}{Aut}   % Automorphismenmenge

\providecommand{\id}{\ensuremath{\textsf{id}}}  % identische Abbildung
\DeclareMathOperator{\Kern}{Kern}     % Kern
\DeclareMathOperator{\Bild}{Bild}     % Bild

\DeclareMathOperator{\ggT}{ggT}       % ggT
\DeclareMathOperator{\kgV}{kgV}       % kgV

\providecommand{\Bin}[2]{\ensuremath{\operatorname{Bin} \left( #1 , #2 \right)}}
\providecommand{\Hyp}[2]{\ensuremath{\operatorname{Hyp} \left( #1 , #2 \right)}}

\providecommand{\PeroW}[2]{\ensuremath{\text{Per}^{#1}_{#2}\text{(oW)}}}
\providecommand{\PermW}[2]{\ensuremath{\text{Per}^{#1}_{#2}\text{(mW)}}}
\providecommand{\KomoW}[2]{\ensuremath{\text{Kom}^{#1}_{#2}\text{(oW)}}}
\providecommand{\KommW}[2]{\ensuremath{\text{Kom}^{#1}_{#2}\text{(mW)}}}

%\usepackage[german, algoruled, linesnumbered, titlenumbered, vlined]{algorithm3e}
%\SetKwInOut{Input}{Eingabe}
%\SetKwInOut{Output}{Ausgabe}
\usepackage{rail}
\railalias{lbrace}{\{}
\railalias{rbrace}{\}}
\railalias{underscore}{\_}
\railalias{dollar}{\$}
\railalias{percent}{\%}
\railalias{ampersand}{<>}
\railalias{backslash}{\char"5C}
\railalias{tilde}{$\sim$}
\railalias{ampersand}{\&}
\railalias{doubleampersand}{\&\&}

\railalias{analyzegraph}{analyze\_graph}
\railalias{gensearchplan}{gen\_searchplan}
\railalias{setmaxmatches}{set\_max\_matches}

\railterm{lbrace,rbrace,dollar,percent,ampersand,backslash,underscore,tilde,ampersand,analyzegraph, gensearchplan,setmaxmatches,doubleampersand}
%\usepackage[noend]{algorithmic}
%\floatname{algorithm}{Algorithmus}
%\renewcommand{\algorithmicrequire}{\textbf{Eingabe:}}
%\renewcommand{\algorithmicensure}{\textbf{Ausgabe:}}

\usepackage{listings}
\lstset{numbers=left, numberstyle=\tiny, breaklines=true}
%\lstset{language=C, numbers=left, numberstyle=\tiny, stepnumber=5}



%\newtheorem{Satz}{Satz}
