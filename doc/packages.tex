\usepackage[utf8]{inputenc}

\usepackage[a4paper, asymmetric, right=2.5cm, left=3.0cm, top=2.5cm, bottom=2.0cm]{geometry}

\usepackage[german, english]{babel}
\usepackage[left]{eurosym}
\usepackage{tabularx}
\newcolumntype{v}[1]{%
  >{\raggedright\hspace{Opt}}p{#1}%
}  
\usepackage{slashbox}

\usepackage{tikz}
\usepackage{pgflibraryarrows}
\usepackage{pgflibraryshapes}
\usepackage{pgflibraryplotmarks}
\usepackage{pgflibrarysnakes}
\usetikzlibrary{backgrounds}
\usepackage{amsmath}
\usepackage{amsfonts}
\usepackage{amssymb}
\usepackage{leftidx}
\usepackage{nicefrac}
\usepackage{thumbpdf}

\usepackage[plainpages=false,
						colorlinks=true, linkcolor=blue, citecolor=blue, urlcolor=red,	
						bookmarks, bookmarksopen, bookmarksopenlevel=1, bookmarksnumbered=true,
						pdftitle={The GrGen User Manual},
						pdfauthor={Jakob Blomer, Rubino Geiss},
						pdfkeywords={
						graph, graph transformation, graph rewriting, tool, spo, manual, grgen, graph rewrite generator,
						University of Karlsruhe, IPD Goos},
						pdfpagelabels
						]{hyperref}

\usepackage[all]{xy}
\usepackage{xcolor}
\definecolor{light}{gray}{0.8}
\definecolor{tuerkis}{cmyk}{0.5,0.15,0,0.3}
\usepackage{graphicx}
\usepackage{calc}

%\usepackage{listings}
%\lstset{language=C, numbers=left, numberstyle=\tiny, stepnumber=5}

\providecommand{\O}[1]{\ensuremath{\mathcal{O}\left( #1 \right)}}

\providecommand{\N}{\ensuremath{\mathbb{N}}}
\providecommand{\Z}{\ensuremath{\mathbb{Z}}}
\renewcommand{\P}{\ensuremath{\mathbb{P}}}
\providecommand{\F}{\ensuremath{\mathbb{F}}}
\providecommand{\Q}{\ensuremath{\mathbb{Q}}}
\providecommand{\R}{\ensuremath{\mathbb{R}}}
\providecommand{\C}{\ensuremath{\mathbb{C}}}

\providecommand{\abs}[1]{\left\lvert #1 \right\rvert}
\providecommand{\norm}[1]{\left\lVert #1 \right\rVert}
\providecommand{\floor}[1]{\left\lfloor #1 \right\rfloor}
\providecommand{\svert}{\; \vert \;}

\DeclareMathOperator{\Abb}{Abb}   % Abbildungsmenge
\DeclareMathOperator{\Sym}{Sym}   % Bijektionenmenge
\DeclareMathOperator{\Hom}{Hom}   % Homomorphismenmenge
\DeclareMathOperator{\End}{End}   % Endomorphismenmenge
\DeclareMathOperator{\Aut}{Aut}   % Automorphismenmenge

\providecommand{\id}{\ensuremath{\textsf{id}}}  % identische Abbildung
\DeclareMathOperator{\Kern}{Kern}     % Kern
\DeclareMathOperator{\Bild}{Bild}     % Bild

\providecommand{\Bin}[2]{\ensuremath{\operatorname{Bin} \left( #1 , #2 \right)}}
\providecommand{\Hyp}[2]{\ensuremath{\operatorname{Hyp} \left( #1 , #2 \right)}}

\providecommand{\PeroW}[2]{\ensuremath{\text{Per}^{#1}_{#2}\text{(oW)}}}
\providecommand{\PermW}[2]{\ensuremath{\text{Per}^{#1}_{#2}\text{(mW)}}}
\providecommand{\KomoW}[2]{\ensuremath{\text{Kom}^{#1}_{#2}\text{(oW)}}}
\providecommand{\KommW}[2]{\ensuremath{\text{Kom}^{#1}_{#2}\text{(mW)}}}

\makeatletter
\def\Relbar{\mathrel{\smash=}}
\def\Leftarrowfill@{\arrowfill@\Leftarrow\Relbar\Relbar}
\def\Rightarrowfill@{\arrowfill@\Relbar\Relbar\Rightarrow}
\newcommand{\xRightarrow}[2][]{\ext@arrow 0359\Rightarrowfill@{#1}{#2}}
\newcommand{\xLeftarrow}[2][]{\ext@arrow 3095\Leftarrowfill@{#1}{#2}}
\makeatother 

\usepackage{rail}
\railalias{lbrace}{\{}
\railalias{rbrace}{\}}
\railalias{underscore}{\_}
\railalias{dollar}{\$}
\railalias{percent}{\%}
\railalias{ampersand}{<>}
\railalias{backslash}{\char"5C}
\railalias{tilde}{$\sim$}
\railalias{ampersand}{\&}
\railalias{doubleampersand}{\&\&}

\railalias{analyzegraph}{analyze\_graph}
\railalias{gensearchplan}{gen\_searchplan}
\railalias{setmaxmatches}{set\_max\_matches}

\railterm{lbrace,rbrace,dollar,percent,ampersand,backslash,underscore,tilde,ampersand,analyzegraph, gensearchplan,setmaxmatches,doubleampersand}

\usepackage{listings}

\usepackage{soul}
\usepackage{titlesec}
\usepackage{titletoc}
\usepackage{fancyhdr}

% sectionreformatting
% tableofcontents level == 3 (subsubsection)
\setcounter{tocdepth}{3}
\setcounter{secnumdepth}{3}

% Headings
\pagestyle{myheadings}

\renewcommand{\chaptermark}[1]{\markboth{\it #1}{}}
\renewcommand{\sectionmark}[1]{\markright{\thesection \; \it #1}}
\lhead[\fancyplain{}{\thepage}]%
      {\fancyplain{}{\rightmark}}
\rhead[\fancyplain{}{\leftmark}]%
  {\fancyplain{}{\thepage}}
\renewcommand{\headrulewidth}{0pt} 

\capsdef{////}{\upshape}{0.125em}{0.4583em}{0.5833em}
\newcommand{\versal}[1]{\MakeUppercase{\caps{#1}}}

% Other desccription labels, not that ugly bold font
% And put the description on a line of its own
\renewcommand\descriptionlabel[1]{%
  \hspace\labelsep {% local changes only
    \advance\linewidth\leftmargin
    \advance\linewidth-\labelsep
    \makebox[\linewidth][l]{\it #1}}}

% Other TOC headings for chapters \contentslabel{2.3em}}
%\titlecontents{chapter}[0em]{\sf\vspace*{2ex}}{Kapitel
%          \thecontentslabel{} --- }
%          {}{\hfill\contentspage\vspace*{1ex}}

\titlecontents{part}[0em]{\sf\large\vspace*{2ex}}{\contentslabel{2.3em}}
              {}{\hfill\contentspage\vspace*{1ex}}

\titlecontents{chapter}[0em]{\sf\vspace*{2ex}}{\contentslabel{2.3em}}
              {}{\hfill\contentspage\vspace*{1ex}}

%\titlecontents{chapter}[1.5em]{\sf}{2.3em}{1pc}{}{}{}
\dottedcontents{section}[3.8em]{}{2.3em}{1pc}

% Other formatting of chapter and section headings
\newcommand*\chaptitle[1]{\Large\filleft\versal{#1}}


\titleformat{\part}[block]
    {\LARGE\sf}
  {\versal{TEIL}\hspace*{2ex}\thepart}
  {4ex}{\Huge\newline\newline\centering}
\titleformat{\chapter}[display]
  {\large\sf}
  {\filright\versal{\chaptertitlename}\hspace*{2ex}\thechapter}
  {4ex}{\chaptitle}
\titleformat{\section}
  {\large\sf}{\thesection}
  {1em}{}
\titleformat{\subsection}
  {\sf}{\thesubsection}
  {1em}{}
\titleformat{\subsubsection}
  {\sf}{\thesubsubsection}
  {1em}{}
\titleformat{\paragraph}
  {\sf}{\theparagraph}{0em}{}
\titlespacing*{\paragraph}{0cm}{2.75ex plus 1ex minus .2ex}{.5em}

\makeatletter
\renewcommand*{\tableofcontents}{%
  \begingroup
    \@restonecolfalse
    \expandafter\chapter\expandafter*\expandafter{\contentsname}%
    \@mkboth{\contentsname}{\contentsname}%
    \@starttoc{toc}%
  \endgroup
} \makeatother


