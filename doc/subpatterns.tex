\chapter{Nested and Subpatterns}\indexmain{rule set language nested and subpatterns}
\label{cha:nestedsub}

The extension of the rule set language described in the previous chapter by nested patterns and subpatterns greatly enhances the flexibility and expressiveness of pattern matching and rewriting.
The following patterns to match a simplified abstract syntax tree give a rough picture of the language of nested and subpatterns:
  \begin{example}
    \begin{grgen}
test method
{
  m:Method <-- n:Name; // signature of method consisting of name
  iterated { // and 0-n parameters
    m <-- :Variable;
  }  
  
  :AssignmentList(m); // body consisting of a list of assignment statements
}

pattern AssignmentList(prev:Node)
{
  optional { // nothing or a linked assignment and again a list
    prev --> a:Assign; // assignment node 
    a -:target-> v:Variable; // which has a variable as target 
    :Expression(a);  // and an expression which defines the left hand side 
    :AssignmentList(a); // next one, plz
  }
}

pattern Expression(root:Expr)
{
  alternative { // expression may be
    Binary { // a binary expression of an operator and two expresions
      root <-- expr1:Expr;
	  :Expression(expr1);
      root <-- expr2:Expr;
	  :Expression(expr2);
      root <-- :Operator;
    }
    Unary { // or a unary expression which is a variable (reading it)
      root <-- v:Variable;
    }
  }
}
    \end{grgen}
  \end{example}\label{introexample}


Until now we have seen rules and tests with one left hand side static pattern specification in a direct 1:1 correspondence with its dynamic match in the host graph on a successful application.
From now on we will increase the expressiveness of the pattern language, and dependent on it the rewrite language, to describe much finer and more flexible what patterns to accept.
This will be done by pattern specifications built up from multiple static pattern piece specifications, where the pieces may be matched dynamically zero, one, or multiple times, or are forbidden to exists for the entire pattern to be matched.
These rule set language constructs can be split into nested patterns (negative application condition, positive application condition, nested pattern with cardinality, alternative patterns) and subpatterns (subpattern declaration and subpattern entity declaration, subrule declaration and usage), we will start with the nested patterns:

\begin{rail}  
  NestedPattern: 
    NegativeApplicationCondition |
    PositiveApplicationCondition |
    NestedPatternWithCardinality |
    AlternativePatterns 
    ;
\end{rail}\ixnterm{NestedPattern}

\section{Negative Application Condition (NAC)}
\indexmain{negative application condition}\indexmainsee{NAC}{negative application condition}\label{nac}

\begin{rail}  
  NegativeApplicationCondition: 
    'negative' lbrace (()+PatternStatement) rbrace;
\end{rail}\ixkeyw{negative}\ixnterm{NegativeApplicationCondition}

With negative application conditions (keyword \texttt{negative}) we can specify graph patterns which forbid the application of a rule if any of them is present in the host graph (cf.~\cite{adam}). 
NACs possess a \indexed{scope} of their own, i.e. names defined within a NAC do not exist outside the NAC. 
Identifiers from surrounding scopes must not be redefined.
If they are not explicitely mentioned, the NAC gets matched independent from them, i.e. elements inside a negative are \texttt{hom(everything from the outside)} by default.
But referencing the element from the outside within the negative pattern causes it to get matched isomorphically/distinct to the other elements in the negative pattern. 
This is a bit unintuitive if you think of extending the pattern by negative elements, but cleaner and more powerful: 
just think of NACs to simply specify a pattern which should not be in the graph, with the possibility of forcing elements to be the same as in the enclosing pattern by name equality.

  \begin{example}
    We specify a variable which is not badly typed, i.e. a node \texttt{x} of type \texttt{Variable} which must not be target of an edge of type \text{type} with a source node of type \texttt{BadType}:
    \begin{grgen}
  x:Variable;
  negative {
    x <-:type- :BadType;
  }
    \end{grgen}
  \end{example}
 
Because NACs have their ``own'' binding, using NACs leads to specifications which might look a bit redundant.

  \begin{example}
    Let's check the singleton condition, meaning there's exactly one node of the type to check, for the type \texttt{RootNamespace}.
    The following specification is \emph{wrong} (it will never return a match):
    \begin{grgen}
  x:RootNamespace;
  negative {
    y:RootNamespace;
  }
    \end{grgen}
	
    Instead we have to specify the \emph{complete} forbidden pattern inside the NAC. This is done by:
    
	\begin{grgen}
  x:RootNamespace;
  negative {
    x;
    y:RootNamespace; // now it is ensured that y must be different from x
  }
    \end{grgen}
	
	Btw: the \texttt{x;} is not a special construct, it's a normal graphlet (cf. \ref{sct:graphlets}).
	
  \end{example} 

If there are several patterns which should not occur, use several negatives.
If there are exceptions to the forbidden pattern, use nested negatives.
As a straight-forward generalization of negatives within positive patterns, negatives may get nested to an arbitrary depth.
Matching of the nested negative pattern causes the matching of the nesting pattern to fail.

\begin{example}
  A fabricated example using parallel as well as nested \texttt{negative}s:
  \begin{grgen}
test onlyOneChildOrAllChildrenHaveExactlyOneCommonChild
{
  root:Class;
  negative {
    root -:extending-> :Class; // root does not extend another class
  }
  root <-:extending- c1:Class; // a class c1 extends root
  negative {
    c1;
    root <-:extending- c2:Class; // there is no c2 which extends root
    negative {
      c1 <-:extending- child:Class -:extending-> c2; // except c1 and c2 have a common child
	  negative { // and c1 has no further children
	    child;
	    c1 <-:extending- :Class;
	  }
	  negative { // and c2 has no further children
	    child;
	    c2 <-:extending- :Class; 
	  }
    }
  }
}
  \end{grgen}
\end{example}

%negative pattern elements get matched independent from the subpatterns utilizing them
%(explicit patternpath/pattern statement in the negative/independent needed for old behaviour)

	
\section{Positive Application Condition (PAC)}
\indexmain{positive application condition}\indexmainsee{PAC}{positive application condition} \label{pac}

\begin{rail}  
  PositiveApplicationCondition: 
    'independent' lbrace (()+PatternStatement) rbrace;
\end{rail}\ixkeyw{independent}\ixnterm{PositiveApplicationCondition}

With positive application conditions (keyword \texttt{independent}) we can specify graph patterns which, in contrast to negative application conditions, must be present in the host graph to cause the matching of the enclosing pattern to succeed.
Together with NACs they share the property of opening a \indexed{scope}, with elements being independent from the surrounding scope (i.e. a host graph element can easiely get matched to a pattern element and a PAC element with a different name, unless the pattern element is referenced in the PAC). 
They are used to improve the logical structure of rules by separating a pure condition from the main pattern of the rule amenable to rewriting.
They are really needed if subpatterns want to match elements which were already matched during the subpattern derivation.

\begin{example}
  A further fabricated example rather giving the intention using \texttt{independent} patterns to check some conditions with only the main pattern available to rewriting:

  \begin{grgen}
rule moveMethod
{
  c:Class --> m:Method;
  csub -:extending-> c;
  csub:Class -e:Edge-> msub:Method;
  
  independent {
    // a complicated pattern to find out that m and msub have same signatures
  }
  independent {
    // a complicated pattern to find out that msub is only using variables available in c
  }
  independent {
    // a complicated pattern to find out that m is not used
  }
 
  modify { // move method upwards
    delete(m);
    delete(e);
    c --> msub;
  }
}
  \end{grgen}
\end{example}

  
\section{Pattern Cardinality}
\indexmainsee{cardinality}{pattern cardinality}\indexmain{pattern cardinality}\label{cardinality}

\begin{rail}  
  NestedPatternWithCardinality: 
    ('iterated' | 'multiple' | 'optional') lbrace NestedBody rbrace;
  NestedBody: (PatternStatement+) NestedRewriting?;
\end{rail}\ixkeyw{iterated}\ixkeyw{multiple}\ixkeyw{optional}\ixnterm{NestedPatternWithCardinality}

The \indexmain{pattern cardinality} blocks allow to specify how often the nested pattern -- opening a scope -- is to be matched.
Matching will be carried out eagerly, i.e. if the construct is not limiting the number of matches and a further match is possible it will be done.
(The nested body will be explained in Section~\ref{sec:nestedrewrite}.)

\subsubsection*{The iterated block} 
is matching the contained subpattern as often as possible, succeeding even in the case the contained pattern is not available (thus it will never fail).
It was included in the language to allow for matching breadth-splitting structures, as in capturing all methods of a class in a program graph.

\begin{example}
  \begin{grgen}
test methods
{
  c:Class;
  iterated {
    c --> m:Method;
  }  
}
  \end{grgen}
\end{example}

\subsubsection*{The multiple block}
is working like the iterated block, but expects the contained subpattern to be available at least once, if it is not, matching of the multiple block and thus its enclosing pattern fails.

\begin{example}
  \begin{grgen}
test oneOrMoreMethods
{
  c:Class;
  multiple {
    c --> m:Method;
  }
}
  \end{grgen}
\end{example}

\subsubsection*{The optional block}
is working like the iterated block, but matches the contained subpattern at most once, further occurences of the subpattern are left unmatched.
If the nested pattern is available, it will get matched, otherwise it won't; matching of the optional block will succeed either way.

\begin{example}
  \begin{grgen}
test variableMaybeInitialized
{
  v:Variable; // match variable
  optional { // and an initialization with a different one if available
    v <-- otherV:Variable;
  }
}
  \end{grgen}
\end{example}

\begin{note}
Pattern cardinality constructs are match/rewrite-all enumeration blockers.
For every pattern instance, the iterated/... yields only one match, even if in all mode.
\end{note} 


\section{Alternative Patterns}
\indexmain{alternative patterns}\label{alternative}

\begin{rail}  
  AlternativePatterns: 
    'alternative' lbrace ((CaseName lbrace NestedBody rbrace)+()) rbrace;
\end{rail}\ixkeyw{alternative}\ixnterm{AlternativePattern}

With the alternative block you can specify several nested alternative patterns. One of them must get matched for the matching of the alternative (and thus its directly nesting pattern) to succeed, and only one of them is matched per match of the alternative / overall pattern.
The order of matching the alternative patterns is unspecified, especially it is not guaranteed that a case gets matched before the case textually following -- if you want to ensure that a case can not get matched if another case could be matched, you must explicitely prevent that from happening by adding negatives to the cases.
In contrast to the iterated which locally matches everything available and inserts this combined match into the current match, the alternative decides for one case match which it inserts into the current match tree, ignoring other possible matches by other cases. 

\begin{example}
  \begin{grgen}
test feature(c:Class)
{
	alternative // a feature of the class is either
	{
		FeatureMethod { // a method
			c --> :Method;
		}
		FeatureVariable { // or a variable
			c --> :Variable;
		}
		FeatureConstant { // or a constant
			c ---> :Constant;
		}
	}
}  
  \end{grgen}
\end{example}

\begin{example}
  \begin{grgen}
test variableMaybeInitialized
{
  v:Variable; // match variable
  alternative { // and an initialization with a different one if available
    Empty {
        // the empty pattern matches always	
	  negative { // so prevent it to match if initialization is available
        v <-- otherV:Variable;
	  }
	}
	Initialized { // initialization
      v <-- otherV:Variable;
	}
  }
}
  \end{grgen}
\end{example}

\section{Subpattern Declaration and Subpattern Entity Declaration}
\indexmain{subpattern}\label{sec:subpattern}

Subpatterns were introduced to factor out a common recurring pattern -- a shape -- into a named subpattern type, ready to be reused at points the pattern should get matched.
The common recurring pattern is specified in a subpattern declaration and used by a subpattern entity declaration.

\begin{rail}  
  SubpatternDeclaration: 
    'pattern' IdentDecl Parameters? (('modify'|'replace') Parameters)? \\
	lbrace SubpatternBody rbrace ;
  SubpatternBody: (PatternStatement+) SubpatternRewriting?;
\end{rail}\ixkeyw{pattern}\ixkeyw{modify}\ixkeyw{replace}\ixnterm{SubpatternDeclaration}\ixnterm{SubpatternBody}\label{subpatterndecl}\indexmain{subpattern declaration}

Subpattern declarations define a subpattern type denoting the specified shape in global namespace; the parameters specify some context elements the pattern may refer to, but which are not part of the pattern itself. 
So they are only syntactically the same as test/rule-parameters (which are members of the pattern part), a further difference is the lack of ReturnTypes, due to the fact that a subpattern can't return anything (they are not actions, just a helper in constructing complex patterns).
Subpatterns can receive additional rewrite parameters in contrast to the actions; they can be used to hand in elements which are created in the rewrite part of the action or subpattern which contains the subpattern entity.
(The nested body will be explained in Section~\ref{sec:subrule}.)

\begin{rail}  
  SubpatternEntityDeclaration: 
    Ident ':' SubpatternType '(' ')' ';' ;
\end{rail}\ixnterm{SubpatternEntityDeclaration}\indexmain{subpattern entity declaration}

Subpattern entity declarations instantiate an entity of the subpattern type (i.e. specified shape), which means the subpattern must get matched for the matching of the enclosing pattern to succeed.
The arguments given are bound to the corresponding parameters of the subpattern.
If you prefer a syntactical point of view, you may see the subpattern entity as a placeholder, which gets substituted in place by the textual body of the subpattern declaration under renaming of the parameters to the arguments.

%TODO:remove
%Termersetzung mit Termen die Graphen beschreiben.
%Mehrelementige Graphsprachen als Zwischenschritt zu den rekursiven bereits mit alternative... eingef�hrt -- dort auch so erkl�rt, wirklich eingef�hrt? vor rekursiv trotzdem noch ein  Zwischenschritt: pattern mit alternative, statt action mit alternative

\begin{example}
  \begin{grgen}
pattern TwoParameters(mp:Method)
{
  mp <-- :Variable;
  mp <-- :Variable;
}
test methodAndFurther
{
  m:Method <-- n:Name;
  tp:TwoParameters(m);
}
  \end{grgen}

  In the given example a subpattern \texttt{TwoParameters} is declared, connecting the context element \texttt{mp} via two edges to two variable nodes.
  The test \texttt{methodAndFurther} is using the subpattern via the declaration of the entity \texttt{tp} of type \texttt{TwoParameters}, binding the context element to its local node \texttt{m}.
  The resulting test after subpattern derivation is equivalent to the test \texttt{methodWithTwoParameters}.

  \begin{grgen}
test methodWithTwoParameters
{
  m:Method <-- n:Name;
  m <-- :Variable;
  m <-- :Variable;
}
  \end{grgen}
\end{example}


\subsection{Recursive Patterns}\indexmain{recursive pattern}\indexmain{structural recursion}

Subpatterns can be combined with alternative patterns or the cardinality patterns into recursive subpatterns, i.e. subpatterns which may contain themselves.
Subpatterns containing themselves alone -- directly or indirectly -- would never yield a match.

\begin{example}
  \begin{grgen}
test iteratedPath
{
  root:Assign;
  negative { --> root; }
  :IteratedPath(root); // match iterated path = assignment list
}

pattern IteratedPath(prev:Node)
{
  optional { // nothing or a linked assignment and again a list
    prev --> a:Assign; // assignment node 
    :IteratedPath(a); // next one, plz
  }
}
  \end{grgen}
  
Searches an iterated path from the root node on, here an assinment list. 
The iterated path with the optional is equivalent to the code below (note the negative which ensures you get a longest match -- without it the empty case may be chosen lazily just in the beginning)
  
  \begin{grgen}
pattern IteratedPath(prev:Node)
{
  alternative {
    Empty {
      negative {
        prev --> a:Assign;
      }
    }
    Further {
      prev --> a:Assign;
      :IteratedPath(a);
    }
  }
}
  \end{grgen}
\end{example}


\begin{example}
  \begin{grgen}
rule removeMiddleAssignment
{
  a1:Assign --> a2:Assign --> a3:Assign;
  independent {
    :IteratedPath(a1,a3)
  }
  
  replace {
    a1; a3;
  }
}

pattern IteratedPath(begin:Assign, end:Assign)
{
  alternative { // an iterated path from begin to end is either
    Found { // the begin assignment directly linked to the end assignment (base case)
      begin --> end;
    }
    Further { // or an iterated path from the node after begin to end (recursive case)
      begin --> intermediate:Assign;
      :IteratedPath(intermediate, end);
    }
  }
}
  \end{grgen}

This is once more a fabricated example, for an iterated path from a source node to a distinctive target node, and an example for the interplay of subpatterns and positive application conditions to chech complex conditions independent from the pattern already matched.
Here, three nodes \texttt{a1},\texttt{a2},\texttt{a3} of type \texttt{Assign} forming a list connected by edges are searched, and if found, \texttt{a2} gets deleted, but only if there is an iterated path of directed edges from \texttt{a1} to \texttt{a3}.
The path may contain the host graph node matched to \texttt{a2} again.
Without the \texttt{independent} this would not be possible, as all pattern elements -- including the ones originating from subpatterns -- get matched isomorphically. 
The same path specified in the pattern of the rule -- not in the independent -- would not get matched if it would go through the host graph node matched to b, as it is locked by the isomorphy constraint.

\end{example}

With recursive subpatterns you can altready capture neatly structures extending into depth (as iterated paths)
and also structures extending into breadth (as forking patterns, although the cardinality statements are often better suited to this task).
But combined with an iterated block, you may even match structures extending into breadth and depth,
like e.g. a hierarchy of classes (i.e. match a spanning tree(todo:indexed) in the graph)
giving you a very powerful and flexible notation to capture large, complex patterns
built up in a stuctured way from simple, connected pieces (as e.g. abstract syntax trees of programming languages).

\begin{example}
  \begin{grgen}
pattern SpanningTree(root:Class)
{
  iterated {
    root <-:extending- next:Class;
    :SpanningTree(next);
  }
}
  \end{grgen}
\end{example}


\section{Nested Pattern Rewriting}
\indexmain{nested pattern rewrite}\label{sec:nestedrewrite}

Until now we focused on the pattern matching of nested and subpatterns -- but we're not only interested in finding patterns combined from several pattern pieces, we want to rewrite the pattern pieces, too.
This does not hold for the application conditions, which are pure conditions, but for all the other language constructs introduced in this chapter.

\begin{rail}  
  NestedRewriting: ('replace' | 'modify') lbrace (()+ReplaceStatement) rbrace;
\end{rail}\ixnterm{NestedRewriting}\ixkeyw{replace}\ixkeyw{modify}

Syntactically the rewrite is specified by a modify or replace clause nested directly within the scope of each nested pattern;
in addition to the rewrite clause nested within the top level pattern, which must be present even if the top level pattern is empty. 
Semantically for every instance of a pattern piece matched its dependent rewrite is applied. 
So in the same manner the complete pattern is assembled from pattern pieces, the complete rewrite gets assembled from rewrite pieces
(or operationally: rewriting is done along the match tree by rewriting one pattern piece after the other).
Note that neither \texttt{exec} statements nor \texttt{return} statements are available as in the top level rewrite part of a rule.

For a static pattern specification like the iterated block yielding dynamically a combined match of zero to many pattern matches, every submatch is rewritten, according to the rewrite specification applied to the host graph elements of the match bound to the pattern elements
(if the pattern was matched zero times, no dependent rewrite will be triggered - but note that zero matches still means success for an iterated, so the dependent rewrite piece of the enclosing pattern will be applied).
This allows e.g. for reversing all edges in the iterated-example (denoting containment in the class), as it is shown in the first of the following two examples.
For the alternative construct the rewrite is specified directly at every nested pattern, i.e. alternative case as shown in the second of the following two examples); the rewrite of the matched case will be applied.

Nodes and edges from the pattern containing the nested pattern containing the nested rewrite are not available for deletion inside the nested rewrite;
only the rewrite part of the pattern in which they are declared is allowed to delete them.
They are available for retyping though, but only in case the retyping is unambigous -- it must be guaranteed that only one retyping can occur.
So only the rewrites of alternative cases and optional patterns may contain retypings of elements not declared in the alternative case or optional pattern (in contrast to iterated and multiple pattern rewrites).

\begin{example}
  \begin{grgen}
rule methods
{
  c:Class;
  iterated {
    c --> m:Method;
	
	replace {
	  c <-- m;
	}
  } 

  replace {
    c;
  }  
}
  \end{grgen}
\end{example}

\begin{example}
  \begin{grgen}
rule methodWithTwoOrThreeParameters(m:Method)
{
  alternative {
    Two {
      m <-- n:Name;
      m <-e1:Edge- v1:Variable;
      m <-e2:Edge- v2:Variable;
	  negative {
	    v1; v2; m <-- :Variable;
	  }
	  
	  modify {
	    delete(e1); m --> v1;
	    delete(e2); m --> v2;	    
	  }
	}
	Three {
      m <-- n:Name;
      m <-e1:Edge- v1:Variable;
      m <-e2:Edge- v2:Variable;
      m <-e3:Edge- v3:Variable;
	  
	  modify {
	    delete(e1); m --> v1;
	    delete(e2); m --> v2;
	    delete(e3); m --> v3;
	  }
	}
	
	modify {
	}
}
  \end{grgen}
\end{example}

\begin{example}
  \begin{grgen}
rule alternativeRelabeling
{
  m:Method;
  
  alternative {
    private {
      if { m.access == Access::private; }
	
      modify {
        pm:PrivateMethod<m>;
      }
	}
	static {
	  negative {
        m <-- c;
	  }
	  
      modify {
        sm:StaticMethod<m>;
      }
	}
  } 

  modify {
  }  
}
  \end{grgen}
\end{example}


\section{Subpattern rewriting}
\indexmain{subrule}\label{sec:subrule}

Alongside the separation into subpattern declaration and subpattern entity declaration, subpattern rewriting is separated into a nested rewrite specification given within the subpattern declaration defining how the rewrite looks like and a subpattern rewrite application given withing the rewrite part of the pattern containing the subpattern entity declaration requesting the rewrite to be actually applied.

\begin{rail}  
  SubpatternRewriting: ('replace' | 'modify') lbrace (()+ReplaceStatement) rbrace;
\end{rail}\ixnterm{SubpatternRewriting}\ixkeyw{replace}\ixkeyw{modify}

The subpattern rewriting specifications within the subpattern declaration looks like a nested rewriting specification,
but additionally there may be rewrite parameters given in the subpattern header (cf. \ref{subpatterndecl}) which can be referenced in the rewrite body.
(Most elements can be handed in with normal parametes, but elements created in the rewrite part of the user of the subpattern can only be handed in at rewrite time.) 

\begin{rail}  
  SubpatternRewriteApplication: 
    Ident '(' ((Ident ':' NodeType)*',') ')' ';' |
    SubpatternOccurence |
	'emithere' '(' StringExpr ')' ';'
	;
\end{rail}\ixnterm{SubpatternRewriteApplication}\ixkeyw{emithere}

The SubpatternRewriteApplication is part of a ReplaceStatement.
The subpattern rewrite application is given within the rewrite part of the pattern containing the subpattern entity declaration, in call notation on the declared subpattern identifier.
The \texttt{emithere}-statements are executed before the emit statements, and in the order in between the subpattern rewrite applications they are specified syntactically.

\begin{example}
  \begin{grgen}
pattern TwoParametersAddDelete(mp:Method)
{
  mp <-- v1:Variable;
  mp <-- :Variable;

  modify {
    delete(v1);
    mp <-- :Variable;
  }
}
rule methodAndFurtherAddDelete
{
  m:Method <-- n:Name;
  tp:TwoParametersAddDelete(m);
  
  modify {
    tp(); // trigger rewriting of the TwoParametersAddDelete instance
  }
}
  \end{grgen}
\end{example}


\begin{example}
  \begin{grgen}
pattern IteratedPathReverse(prev:Node)
{
  optional {
    prev --> next:Node;
    ipr:IteratedPathReverse(next);
    
    replace {
    	prev <-- next;
    	ipr();
    }
  }
  
  replace {
  }
}
  \end{grgen}
  Reversing the direction of the edges in an iterated path.
\end{example}

\begin{example}
This is an example for rewrite parameters, connecting every node on an iterated path to a common node (i.e. the local rewrite graph to the containing rewrite graph).
It can't be simulated by subpattern parameters which get defined at matching time because the common element is only created later on, at rewrite time.

  \begin{grgen}
pattern ChainFromToReverseToCommon(from:Node, to:Node) replace(common:Node)
{
  alternative {
    rec {
      from --> intermediate:Node;
      cftrtc:ChainFromToReverseToCommon(intermediate, to);

      replace {
        from <-- intermediate;
        from --> common;
        cftrtc(common);
      }
    }
	base {
      from --> to;

      replace {
        from <-- to;
        from --> common;
        to --> common;
      }
    }
  }

  replace {
    from; to;
  }
}
  \end{grgen}

  \begin{grgen}  
  rule chainFromToReverseToCommon()
  {
    from:Node; to:Node;
	cftrtc:ChainFromToReverseToCommon();
	
	modify {
	  common:Node;
	  cftrtc(common);
	}
  }
  \end{grgen}
\end{example}


\subsection{Deletion and Preservation of Subpatterns}\label{sub:delpressub}

In addition to the fine-grain dependent replacement, subpatterns may get deleted or kept as a whole.

\begin{rail}  
  SubpatternOccurence: 
    Ident ';' |
    'delete' '(' (Ident + ',') ')' ';';
\end{rail}\ixkeyw{SubpatternOccurence}\ixkeyw{delete}

In modify mode, they are kept by default, but deleted if the name of the declared subpattern entity is mentioned within a delete statement.
In replace mode, they are deleted by default, but kept if the name of the declared subpattern entity is mentioned (using occurence, same as with nodes or edges).

\begin{example}
  \begin{grgen}
rule R {
  m1:Method; m2:Method;
  tp1:TwoParameters(m1);
  tp2:TwoParameters(m2);

  replace {
    tp1; // is kept
    // tp2 not included here - will be deleted
    // tp1(); or tp2(); -- would apply dependent replacement
    m1; m2;
  }
}
  \end{grgen}
\end{example}


\section{Regular Expression Syntax}\indexmainsee{EBNF}{regular expression syntax}\indexmain{regular expression syntax}

In addition to the already introduced syntax for the nested patterns with the keywords 
\texttt{negative}, \texttt{independent}, \texttt{alternative}, \texttt{iterated}, \texttt{multiple} and \texttt{optional},
there is a more lightweight syntax resembling regular expressions; 
using it together with the subpatterns gives graph rewrite specifications which look like EBNF-grammars with embedded actions. 
Exceeding the more verbose syntax they offer constructs for matching the pattern a bounded number of times (same notation as the one for the bounded iteration in the xgrs).
Table \ref{keywordregexpsyntax} lists the corresponding (/equivalent) language constructs; 
Example \ref{introexampleregexp} is a version of the introductionary example \ref{introexample} modified to use the new syntax.

\begin{table}[htbp]
  \centering
  \begin{tabularx}{\linewidth}{|l|X|} \hline
    \texttt{ iterated \{ P \} } & \texttt{ (P)* } \\
    \texttt{ multiple \{ P \} } & \texttt{ (P)+ } \\
    \texttt{ optional \{ P \} } & \texttt{ (P)? } \\ 
	\texttt{ alternative \{ l1 \{ P1 \} .. lk \{ Pk \} \} } & \texttt{ (P1|..|Pk) } \\
    \texttt{ negative \{ P \} } & \texttt{ $\sim$(P) } \\
    \texttt{ independent \{ P \} } & \texttt{ \&(P)} \\ \hline 
    \texttt{ modify \{ R \} } & \texttt{  \{+ R \} } \\
    \texttt{ replace \{ R \} } & \texttt{ \{- R \} } \\ \hline 
    \texttt{ - } & \texttt{ (P)[k] / (P)[k:l] / (P)[k:*] } \\ \hline 
	\end{tabularx}
  \caption{Map of nested patterns in keyword syntax to regular expression syntax}
  \label{keywordregexpsyntax}
\end{table}


  \begin{example}
    \begin{grgen}
test method
{
  m:Method <-- n:Name; // signature of method consisting of name
  ( m <-- :Variable; )* // and 0-n parameters
  
  :AssignmentList(m); // body consisting of a list of assignment statements
}

pattern AssignmentList(prev:Node)
{
  ( // nothing or a linked assignment and again a list
    prev --> a:Assign; // assignment node 
    a -:target-> v:Variable; // which has a variable as target 
    :Expression(a);  // and an expression which defines the left hand side 
    :AssignmentList(a); // next one, plz
  )?
}

pattern Expression(root:Expr)
{
  ( // expression may be a binary expression of an operator and two expresions
      root <-- expr1:Expr;
	  :Expression(expr1);
      root <-- expr2:Expr;
	  :Expression(expr2);
      root <-- :Operator;
  | // or a unary expression which is a variable (reading it)
      root <-- v:Variable;
  )
}
    \end{grgen}
  \end{example}\label{introexampleregexp}



