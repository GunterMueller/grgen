\chapter{Nested and Subpatterns}\indexmain{rule set language nested and subpatterns}
\label{cha:nestedsub}

The extension of the rule set language described in the previous chapter by nested patterns and subpatterns greatly enhances the flexibility and expressiveness of pattern matching and rewriting.
The following patterns to match a simplified abstract syntax tree give a rough picture of the language of nested and subpatterns:
  \begin{example}
    \begin{grgen}
test method
{
  m:Method <-- n:Name; // signature of method consisting of name
  iterated { // and 0-n parameters
    m <-- :Variable;
  }  
  
  :AssignmentList(m); // body consisting of a list of assignment statements
}

pattern AssignmentList(prev:Node)
{
  optional { // nothing or a linked assignment and again a list
    prev --> a:Assign; // assignment node 
    a -:target-> v:Variable; // which has a variable as target 
    :Expression(a);  // and an expression which defines the left hand side 
    :AssignmentList(a); // next one, plz
  }
}

pattern Expression(root:Expr)
{
  alternative { // expression may be
    Binary { // a binary expression of an operator and two expresions
      root <-- expr1:Expr;
	  :Expression(expr1);
      root <-- expr2:Expr;
	  :Expression(expr2);
      root <-- :Operator;
    }
    Unary { // or a unary expression which is a variable (reading it)
      root <-- v:Variable;
    }
  }
}
    \end{grgen}
  \end{example}\label{introexample}


Until now we have seen rules and tests with one left hand side static pattern specification in a direct 1:1 correspondence with its dynamic match in the host graph on a successful application.
From now on we will increase the expressiveness of the pattern language, and dependent on it the rewrite language, to describe much finer and more flexible what patterns to accept.
This will be done by pattern specifications built up from multiple static pattern piece specifications, where the pieces may be matched dynamically zero, one, or multiple times, or are forbidden to exists for the entire pattern to be matched.
These rule set language constructs can be split into nested patterns (negative application condition, positive application condition, nested pattern with cardinality, alternative patterns) and subpatterns (subpattern declaration and subpattern entity declaration, subrule declaration and usage), we will start with the nested patterns:

\begin{rail}  
  NestedPattern: 
    NegativeApplicationCondition |
    PositiveApplicationCondition |
    NestedPatternWithCardinality |
    AlternativePatterns 
    ;
\end{rail}\ixnterm{NestedPattern}


%%%%%%%%%%%%%%%%%%%%%%%%%%%%%%%%%%%%%%%%%%%%%%%%%%%%%%%%%%%%%%%%%%%%%%%%%%%%%%%%%%%%%%%%%%%%%%%%
\section{Negative Application Condition (NAC)}
\indexmain{negative application condition}\indexmainsee{NAC}{negative application condition}\label{nac}

\begin{rail}  
  NegativeApplicationCondition: 
    'negative' lbrace (()+PatternStatement) rbrace;
\end{rail}\ixkeyw{negative}\ixnterm{NegativeApplicationCondition}

With negative application conditions (keyword \texttt{negative}) we can specify graph patterns which forbid the application of a rule if any of them is present in the host graph (cf.~\cite{adam}). 
NACs possess a \indexed{scope} of their own, i.e. names defined within a NAC do not exist outside the NAC. 
Identifiers from surrounding scopes must not be redefined.
If they are not explicitely mentioned, the NAC gets matched independent from them, i.e. elements inside a negative are \texttt{hom(everything from the outside)} by default.
But referencing the element from the outside within the negative pattern causes it to get matched isomorphically/distinct to the other elements in the negative pattern. 
This is a bit unintuitive if you think of extending the pattern by negative elements, but cleaner and more powerful: 
just think of NACs to simply specify a pattern which should not be in the graph, with the possibility of forcing elements to be the same as in the enclosing pattern by name equality.

  \begin{example}
    We specify a variable which is not badly typed, i.e. a node \texttt{x} of type \texttt{Variable} which must not be target of an edge of type \text{type} with a source node of type \texttt{BadType}:
    \begin{grgen}
  x:Variable;
  negative {
    x <-:type- :BadType;
  }
    \end{grgen}
  \end{example}
 
Because NACs have their ``own'' binding, using NACs leads to specifications which might look a bit redundant.

  \begin{example}
    Let's check the singleton condition, meaning there's exactly one node of the type to check, for the type \texttt{RootNamespace}.
    The following specification is \emph{wrong} (it will never return a match):
    \begin{grgen}
  x:RootNamespace;
  negative {
    y:RootNamespace;
  }
    \end{grgen}
	
    Instead we have to specify the \emph{complete} forbidden pattern inside the NAC. This is done by:
    
	\begin{grgen}
  x:RootNamespace;
  negative {
    x;
    y:RootNamespace; // now it is ensured that y must be different from x
  }
    \end{grgen}
	
	Btw: the \texttt{x;} is not a special construct, it's a normal graphlet (cf. \ref{sct:graphlets}).
	
  \end{example} 

If there are several patterns which should not occur, use several negatives.
If there are exceptions to the forbidden pattern, use nested negatives.
As a straight-forward generalization of negatives within positive patterns, negatives may get nested to an arbitrary depth.
Matching of the nested negative pattern causes the matching of the nesting pattern to fail.

\begin{example}
  A fabricated example using parallel as well as nested \texttt{negative}s:
  \begin{grgen}
test onlyOneChildOrAllChildrenHaveExactlyOneCommonChild
{
  root:Class;
  negative {
    root -:extending-> :Class; // root does not extend another class
  }
  root <-:extending- c1:Class; // a class c1 extends root
  negative {
    c1;
    root <-:extending- c2:Class; // there is no c2 which extends root
    negative {
      c1 <-:extending- child:Class -:extending-> c2; // except c1 and c2 have a common child
      negative { // and c1 has no further children
        child;
        c1 <-:extending- :Class;
      }
      negative { // and c2 has no further children
        child;
        c2 <-:extending- :Class; 
      }
    }
  }
}
  \end{grgen}
\end{example}

%negative pattern elements get matched independent from the subpatterns utilizing them
%(explicit patternpath/pattern statement in the negative/independent needed for old behaviour)


%%%%%%%%%%%%%%%%%%%%%%%%%%%%%%%%%%%%%%%%%%%%%%%%%%%%%%%%%%%%%%%%%%%%%%%%%%%%%%%%%%%%%%%%%%%%%%%%
\section{Positive Application Condition (PAC)}
\indexmain{positive application condition}\indexmainsee{PAC}{positive application condition} \label{pac}

\begin{rail}  
  PositiveApplicationCondition: 
    'independent' lbrace (()+PatternStatement) rbrace;
\end{rail}\ixkeyw{independent}\ixnterm{PositiveApplicationCondition}

With positive application conditions (keyword \texttt{independent}) we can specify graph patterns which, in contrast to negative application conditions, must be present in the host graph to cause the matching of the enclosing pattern to succeed.
Together with NACs they share the property of opening a \indexed{scope}, with elements being independent from the surrounding scope (i.e. a host graph element can easily get matched to a pattern element and a PAC element with a different name, unless the pattern element is referenced in the PAC). 
They are used to improve the logical structure of rules by separating a pure condition from the main pattern of the rule amenable to rewriting.
They are really needed if subpatterns want to match elements which were already matched during the subpattern derivation.

\begin{example}
  A further fabricated example rather giving the intention using \texttt{independent} patterns to check some conditions with only the main pattern available to rewriting:

  \begin{grgen}
rule moveMethod
{
  c:Class --> m:Method;
  csub -:extending-> c;
  csub:Class -e:Edge-> msub:Method;
  
  independent {
    // a complicated pattern to find out that m and msub have same signatures
  }
  independent {
    // a complicated pattern to find out that msub is only using variables available in c
  }
  independent {
    // a complicated pattern to find out that m is not used
  }
 
  modify { // move method upwards
    delete(m);
    delete(e);
    c --> msub;
  }
}
  \end{grgen}
\end{example}

  
%%%%%%%%%%%%%%%%%%%%%%%%%%%%%%%%%%%%%%%%%%%%%%%%%%%%%%%%%%%%%%%%%%%%%%%%%%%%%%%%%%%%%%%%%%%%%%%%
\section{Pattern Cardinality}
\indexmainsee{cardinality}{pattern cardinality}\indexmain{pattern cardinality}\label{cardinality}

\begin{rail}  
  NestedPatternWithCardinality: 
    ('iterated' | 'multiple' | 'optional') lbrace NestedBody rbrace;
  NestedBody: (PatternStatement+) NestedRewriting?;
\end{rail}\ixkeyw{iterated}\ixkeyw{multiple}\ixkeyw{optional}\ixnterm{NestedPatternWithCardinality}

The \indexmain{pattern cardinality} blocks allow to specify how often the nested pattern -- opening a scope -- is to be matched.
Matching will be carried out eagerly, i.e. if the construct is not limiting the number of matches and a further match is possible it will be done.
(The nested body will be explained in Section~\ref{sec:nestedrewrite}.)

\subsubsection*{The iterated block} 
is matching the contained subpattern as often as possible, succeeding even in the case the contained pattern is not available (thus it will never fail).
It was included in the language to allow for matching breadth-splitting structures, as in capturing all methods of a class in a program graph.

\begin{example}
  \begin{grgen}
test methods
{
  c:Class;
  iterated {
    c --> m:Method;
  }  
}
  \end{grgen}
\end{example}

\subsubsection*{The multiple block}
is working like the iterated block, but expects the contained subpattern to be available at least once; if it is not, matching of the multiple block and thus its enclosing pattern fails.

\begin{example}
  \begin{grgen}
test oneOrMoreMethods
{
  c:Class;
  multiple {
    c --> m:Method;
  }
}
  \end{grgen}
\end{example}

\subsubsection*{The optional block}
is working like the iterated block, but matches the contained subpattern at most once; further occurences of the subpattern are left unmatched.
If the nested pattern is available, it will get matched, otherwise it won't; matching of the optional block will succeed either way.

\begin{example}
  \begin{grgen}
test variableMaybeInitialized
{
  v:Variable; // match variable
  optional { // and an initialization with a different one if available
    v <-- otherV:Variable;
  }
}
  \end{grgen}
\end{example}

\begin{note}
Pattern cardinality constructs are match/rewrite-all enumeration blockers.
For every pattern instance, the iterated/... yields only one match, even if in all mode (used in/from all-bracketed rules).
\end{note} 


%%%%%%%%%%%%%%%%%%%%%%%%%%%%%%%%%%%%%%%%%%%%%%%%%%%%%%%%%%%%%%%%%%%%%%%%%%%%%%%%%%%%%%%%%%%%%%%%
\section{Alternative Patterns}
\indexmain{alternative patterns}\label{alternative}

\begin{rail}  
  AlternativePatterns: 
    'alternative' lbrace ((CaseName lbrace NestedBody rbrace)+()) rbrace;
\end{rail}\ixkeyw{alternative}\ixnterm{AlternativePattern}

With the alternative block you can specify several nested alternative patterns. One of them must get matched for the matching of the alternative (and thus its directly nesting pattern) to succeed, and only one of them is matched per match of the alternative / overall pattern.
The order of matching the alternative patterns is unspecified, especially it is not guaranteed that a case gets matched before the case textually following -- if you want to ensure that a case cannot get matched if another case could be matched, you must explicitely prevent that from happening by adding negatives to the cases.
In contrast to the iterated which locally matches everything available and inserts this combined match into the current match, the alternative decides for one case match which it inserts into the current match tree, ignoring other possible matches by other cases. 

\begin{example}
  \begin{grgen}
test feature(c:Class)
{
	alternative // a feature of the class is either
	{
		FeatureMethod { // a method
			c --> :Method;
		}
		FeatureVariable { // or a variable
			c --> :Variable;
		}
		FeatureConstant { // or a constant
			c ---> :Constant;
		}
	}
}  
  \end{grgen}
\end{example}

\begin{example}
  \begin{grgen}
test variableMaybeInitialized
{
  v:Variable; // match variable
  alternative { // and an initialization with a different one if available
    Empty {
        // the empty pattern matches always	
	  negative { // so prevent it to match if initialization is available
        v <-- otherV:Variable;
	  }
	}
	Initialized { // initialization
      v <-- otherV:Variable;
	}
  }
}
  \end{grgen}
\end{example}


%%%%%%%%%%%%%%%%%%%%%%%%%%%%%%%%%%%%%%%%%%%%%%%%%%%%%%%%%%%%%%%%%%%%%%%%%%%%%%%%%%%%%%%%%%%%%%%%
\section{Subpattern Declaration and Subpattern Entity Declaration}
\indexmain{subpattern}\label{sec:subpattern}

Subpatterns were introduced to factor out a common recurring pattern -- a shape -- into a named subpattern type, ready to be reused at points the pattern should get matched.
The common recurring pattern is specified in a subpattern declaration and used by a subpattern entity declaration.

\begin{rail}  
  SubpatternDeclaration: 
    'pattern' IdentDecl Parameters? (('modify'|'replace') Parameters)? \\
	lbrace SubpatternBody rbrace ;
  SubpatternBody: (PatternStatement+) SubpatternRewriting?;
\end{rail}\ixkeyw{pattern}\ixkeyw{modify}\ixkeyw{replace}\ixnterm{SubpatternDeclaration}\ixnterm{SubpatternBody}\label{subpatterndecl}\indexmain{subpattern declaration}

Subpattern declarations define a subpattern type denoting the specified shape in global namespace; the parameters specify some context elements the pattern may refer to, but which are not part of the pattern itself. 
So they are only syntactically the same as test/rule-parameters (which are members of the pattern part), a further difference is the lack of ReturnTypes, due to the fact that a subpattern can't return anything (they are not actions, just a helper in constructing complex patterns).
Subpatterns can receive additional rewrite parameters in contrast to the actions; they can be used to hand in elements which are created in the rewrite part of the action or subpattern which contains the subpattern entity.
(The nested body will be explained in Section~\ref{sec:subrule}.)

\begin{rail}  
  SubpatternEntityDeclaration: 
    Ident ':' SubpatternType '(' ')' ';' ;
\end{rail}\ixnterm{SubpatternEntityDeclaration}\indexmain{subpattern entity declaration}

Subpattern entity declarations instantiate an entity of the subpattern type (i.e. specified shape), which means the subpattern must get matched for the matching of the enclosing pattern to succeed.
The arguments given are bound to the corresponding parameters of the subpattern.
If you prefer a syntactical point of view, you may see the subpattern entity as a placeholder, which gets substituted in place by the textual body of the subpattern declaration under renaming of the parameters to the arguments.

%TODO:remove
%Termersetzung mit Termen die Graphen beschreiben.
%Mehrelementige Graphsprachen als Zwischenschritt zu den rekursiven bereits mit alternative... eingef�hrt -- dort auch so erkl�rt, wirklich eingef�hrt? vor rekursiv trotzdem noch ein  Zwischenschritt: pattern mit alternative, statt action mit alternative

\begin{example}
  \begin{grgen}
pattern TwoParameters(mp:Method)
{
  mp <-- :Variable;
  mp <-- :Variable;
}
test methodAndFurther
{
  m:Method <-- n:Name;
  tp:TwoParameters(m);
}
  \end{grgen}

  In the given example a subpattern \texttt{TwoParameters} is declared, connecting the context element \texttt{mp} via two edges to two variable nodes.
  The test \texttt{methodAndFurther} is using the subpattern via the declaration of the entity \texttt{tp} of type \texttt{TwoParameters}, binding the context element to its local node \texttt{m}.
  The resulting test after subpattern derivation is equivalent to the test \texttt{methodWithTwoParameters}.

  \begin{grgen}
test methodWithTwoParameters
{
  m:Method <-- n:Name;
  m <-- :Variable;
  m <-- :Variable;
}
  \end{grgen}
\end{example}

%-----------------------------------------------------------------------------------------------
\subsection{Recursive Patterns}\indexmain{recursive pattern}\indexmain{structural recursion}

Subpatterns can be combined with alternative patterns or the cardinality patterns into recursive subpatterns, i.e. subpatterns which may contain themselves.
Subpatterns containing themselves alone -- directly or indirectly -- would never yield a match.

\begin{example}
  \begin{grgen}
test iteratedPath
{
  root:Assign;
  negative { --> root; }
  :IteratedPath(root); // match iterated path = assignment list
}

pattern IteratedPath(prev:Node)
{
  optional { // nothing or a linked assignment and again a list
    prev --> a:Assign; // assignment node 
    :IteratedPath(a); // next one, plz
  }
}
  \end{grgen}
  
Searches an iterated path from the root node on, here an assinment list. 
The iterated path with the optional is equivalent to the code below (note the negative which ensures you get a longest match -- without it the empty case may be chosen lazily just in the beginning)
  
  \begin{grgen}
pattern IteratedPath(prev:Node)
{
  alternative {
    Empty {
      negative {
        prev --> a:Assign;
      }
    }
    Further {
      prev --> a:Assign;
      :IteratedPath(a);
    }
  }
}
  \end{grgen}
\end{example}


\begin{example}
  \begin{grgen}
rule removeMiddleAssignment
{
  a1:Assign --> a2:Assign --> a3:Assign;
  independent {
    :IteratedPath(a1,a3)
  }
  
  replace {
    a1; a3;
  }
}

pattern IteratedPath(begin:Assign, end:Assign)
{
  alternative { // an iterated path from begin to end is either
    Found { // the begin assignment directly linked to the end assignment (base case)
      begin --> end;
    }
    Further { // or an iterated path from the node after begin to end (recursive case)
      begin --> intermediate:Assign;
      :IteratedPath(intermediate, end);
    }
  }
}
  \end{grgen}

This is once more a fabricated example, for an iterated path from a source node to a distinctive target node, and an example for the interplay of subpatterns and positive application conditions to chech complex conditions independent from the pattern already matched.
Here, three nodes \texttt{a1},\texttt{a2},\texttt{a3} of type \texttt{Assign} forming a list connected by edges are searched, and if found, \texttt{a2} gets deleted, but only if there is an iterated path of directed edges from \texttt{a1} to \texttt{a3}.
The path may contain the host graph node matched to \texttt{a2} again.
Without the \texttt{independent} this would not be possible, as all pattern elements -- including the ones originating from subpatterns -- get matched isomorphically. 
The same path specified in the pattern of the rule -- not in the independent -- would not get matched if it would go through the host graph node matched to b, as it is locked by the isomorphy constraint.

\end{example}

With recursive subpatterns you can already capture neatly structures extending into depth (as iterated paths)
and also structures extending into breadth (as forking patterns, although the cardinality statements are often better suited to this task).
But combined with an iterated block, you may even match structures extending into breadth and depth,
like e.g. a hierarchy of classes (i.e. match a \indexed{spanning tree} in the graph)
giving you a very powerful and flexible notation to capture large, complex patterns
built up in a structured way from simple, connected pieces (as e.g. abstract syntax trees of programming languages).

\begin{note}
If you are working with hierarchic structures like that, 
you might be intersted in the capablities of GrShell/yComp for nested layout
as described and shown in \ref{sub:visual}/\ref{note:visual}).
\end{note}

\begin{example}
  \begin{grgen}
pattern SpanningTree(root:Class)
{
  iterated {
    root <-:extending- next:Class;
    :SpanningTree(next);
  }
}
  \end{grgen}
\end{example}


%%%%%%%%%%%%%%%%%%%%%%%%%%%%%%%%%%%%%%%%%%%%%%%%%%%%%%%%%%%%%%%%%%%%%%%%%%%%%%%%%%%%%%%%%%%%%%%%
\section{Nested Pattern Rewriting}
\indexmain{nested pattern rewrite}\label{sec:nestedrewrite}

Until now we focused on the pattern matching of nested and subpatterns -- but we're not only interested in finding patterns combined from several pattern pieces, we want to rewrite the pattern pieces, too.
This does not hold for the application conditions, which are pure conditions, but for all the other language constructs introduced in this chapter.

\begin{rail}  
  NestedRewriting: ('replace' | 'modify') lbrace (()+ReplaceStatement) rbrace;
\end{rail}\ixnterm{NestedRewriting}\ixkeyw{replace}\ixkeyw{modify}

Syntactically the rewrite is specified by a modify or replace clause nested directly within the scope of each nested pattern;
in addition to the rewrite clause nested within the top level pattern, which must be present even if the top level pattern is empty. 
Semantically for every instance of a pattern piece matched its dependent rewrite is applied. 
So in the same manner the complete pattern is assembled from pattern pieces, the complete rewrite gets assembled from rewrite pieces
(or operationally: rewriting is done along the match tree by rewriting one pattern piece after the other).
Note that \texttt{return} statements are not available as in the top level rewrite part of a rule, and the \texttt{exec} statements are slightly different.

For a static pattern specification like the iterated block yielding dynamically a combined match of zero to many pattern matches, every submatch is rewritten, according to the rewrite specification applied to the host graph elements of the match bound to the pattern elements
(if the pattern was matched zero times, no dependent rewrite will be triggered - but note that zero matches still means success for an iterated, so the dependent rewrite piece of the enclosing pattern will be applied).
This allows e.g. for reversing all edges in the iterated-example (denoting containment in the class), as it is shown in the first of the following two examples.
For the alternative construct the rewrite is specified directly at every nested pattern, i.e. alternative case as shown in the second of the following two examples); the rewrite of the matched case will be applied.

Nodes and edges from the pattern containing the nested pattern containing the nested rewrite are only available for deletion or retyping inside the nested rewrite if it can be statically determined this is unambigous, i.e. only happening once.
So only the rewrites of alternative cases, optional patterns or subpatterns may contain deletions or retypings of elements not declared in their pattern (in contrast to iterated and multiple pattern rewrites).

\begin{example}
%This is an example for a rewrite part nested within an iterated block. - without the comment the two examples fit on one page
  \begin{grgen}
rule methods
{
  c:Class;
  iterated {
    c --> m:Method;
	
	replace {
	  c <-- m;
	}
  } 

  replace {
    c;
  }  
}
  \end{grgen}
\end{example}

\begin{example}
%This is an example for a rewrite parts nested within alternative cases. - without the comment the two examples fit on one page
  \begin{grgen}
rule methodWithTwoOrThreeParameters(m:Method)
{
  alternative {
    Two {
      m <-- n:Name;
      m <-e1:Edge- v1:Variable;
      m <-e2:Edge- v2:Variable;
	  negative {
	    v1; v2; m <-- :Variable;
	  }
	  
	  modify {
	    delete(e1); m --> v1;
	    delete(e2); m --> v2;	    
	  }
	}
	Three {
      m <-- n:Name;
      m <-e1:Edge- v1:Variable;
      m <-e2:Edge- v2:Variable;
      m <-e3:Edge- v3:Variable;
	  
	  modify {
	    delete(e1); m --> v1;
	    delete(e2); m --> v2;
	    delete(e3); m --> v3;
	  }
	}
	
	modify {
	}
}
  \end{grgen}
\end{example}

\begin{example}
This is an example which shows how to decide with an alternative on the target type of a retyping depending on the context.

  \begin{grgen}
rule alternativeRelabeling
{
  m:Method;
  
  alternative {
    private {
      if { m.access == Access::private; }
	
      modify {
        pm:PrivateMethod<m>;
      }
	  }
	  static {
	    negative {
        m <-- c;
	    }
	  
      modify {
        sm:StaticMethod<m>;
      }
	  }
  } 

  modify {
  }
}
  \end{grgen}
\end{example}


%%%%%%%%%%%%%%%%%%%%%%%%%%%%%%%%%%%%%%%%%%%%%%%%%%%%%%%%%%%%%%%%%%%%%%%%%%%%%%%%%%%%%%%%%%%%%%%%
\section{Subpattern rewriting}
\indexmain{subrule}\label{sec:subrule}

Alongside the separation into subpattern declaration and subpattern entity declaration, 
subpattern rewriting is separated into a nested rewrite specification given within the subpattern declaration defining how the rewrite looks like 
and a subpattern rewrite application given within the rewrite part of the pattern containing the subpattern entity declaration requesting the rewrite to be actually applied.

\begin{rail}  
  SubpatternRewriting: ('replace' | 'modify') lbrace (()+ReplaceStatement) rbrace;
\end{rail}\ixnterm{SubpatternRewriting}\ixkeyw{replace}\ixkeyw{modify}

The subpattern rewriting specifications within the subpattern declaration looks like a nested rewriting specification,
but additionally there may be rewrite parameters given in the subpattern header (cf. \ref{subpatterndecl}) which can be referenced in the rewrite body.
(Most elements can be handed in with normal parameters, but elements created in the rewrite part of the user of the subpattern can only be handed in at rewrite time.) 

\begin{rail}  
  SubpatternRewriteApplication: 
    Ident '(' ((Ident ':' NodeType)*',') ')' ';' |
    SubpatternOccurence |
	'emithere' '(' StringExpr ')' ';' |
	'exec' '(' RewriteSequence ')' ';'
	;
\end{rail}\ixnterm{SubpatternRewriteApplication}\ixkeyw{emithere}

\noindent The \emph{SubpatternRewriteApplication} is part of the \emph{ReplaceStatement} already introduced (cf. \ref{replstmt}).
The subpattern rewrite application is given within the rewrite part of the pattern containing the subpattern entity declaration,
in call notation on the declared subpattern identifier.
It causes the rewrite part of the subpattern to get used; if you leave it out, the subpattern is simply kept untouched.
The \emph{SubpatternOccurence} is explained in the next subsection \ref{sub:delpressub}.
The \texttt{emithere}-statements are executed before the emit statements,
and in the order in between the subpattern rewrite applications they are specified syntactically.
The \texttt{exec}-statements listed here are a slightly different version of the \texttt{exec}-statements from the \emph{ExecStatement} introduced in \ref{replclause}.
They are only available in the rewrite parts of subpatterns or nested alternatives/iterateds,
but not in the rewrite part of rules as the original embedded sequences.
They are executed after the original rule calling them was executed,
so they can't get extended by \texttt{yield}s, 
as the containing rule is not available any more when they get executed.

\begin{note}
The embedded sequences are executed after the top-level rule which contains them in a nested pattern or in a used  subpattern was executed; they are \emph{not} executed during subpattern rewriting. They allow you to put work you can't do while executing the rule proper (e.g. because an element was already matched and is now locked due to the isomorphy constraint) to a waiting queue which gets processed afterwards --- with access to the elements of the rule and contained parts which are available when the rule gets executed. Or to just split the work into several parts, reusing already available functionality, see \ref{sub:copymergesplit} for a discussion on this topic.
\end{note}

\begin{example}
This is an example for a subpattern rewrite application.

  \begin{grgen}
pattern TwoParametersAddDelete(mp:Method)
{
  mp <-- v1:Variable;
  mp <-- :Variable;

  modify {
    delete(v1);
    mp <-- :Variable;
  }
}
rule methodAndFurtherAddDelete
{
  m:Method <-- n:Name;
  tp:TwoParametersAddDelete(m);
  
  modify {
    tp(); // trigger rewriting of the TwoParametersAddDelete instance
  }
}
  \end{grgen}
\end{example}


\begin{example}
This is another example for a subpattern rewrite application,
reversing the direction of the edges on an iterated path.

  \begin{grgen}
pattern IteratedPathReverse(prev:Node)
{
  optional {
    prev --> next:Node;
    ipr:IteratedPathReverse(next);
    
    replace {
    	prev <-- next;
    	ipr();
    }
  }
  
  replace {
  }
}
  \end{grgen}
\end{example}

\begin{example}
This is an example for rewrite parameters, connecting every node on an iterated path to a common node (i.e. the local rewrite graph to the containing rewrite graph).
It can't be simulated by subpattern parameters which get defined at matching time because the common element is only created later on, at rewrite time.

  \begin{grgen}
pattern ChainFromToReverseToCommon(from:Node, to:Node) replace(common:Node)
{
	alternative {
		rec {
			from --> intermediate:Node;
			cftrtc:ChainFromToReverseToCommon(intermediate, to);

			replace {
				from <-- intermediate;
				from --> common;
				cftrtc(common);
			}
		}
		base {
			from --> to;

			replace {
				from <-- to;
				from --> common;
				to --> common;
			}
		}
	}

	replace {
		from; to;
	}
}
  \end{grgen}

  \begin{grgen}  
rule chainFromToReverseToCommon()
{
	from:Node; to:Node;
	cftrtc:ChainFromToReverseToCommon(from, to);

	modify {
		common:Node;
		cftrtc(common);
	}
}
  \end{grgen}
\end{example}

%-----------------------------------------------------------------------------------------------
\subsection{Deletion and Preservation of Subpatterns}\label{sub:delpressub}

In addition to the fine-grain dependent replacement, subpatterns may get deleted or kept as a whole.

\begin{rail}  
  SubpatternOccurence: 
    Ident ';' |
    'delete' '(' (Ident + ',') ')' ';';
\end{rail}\ixkeyw{SubpatternOccurence}\ixkeyw{delete}

In modify mode, they are kept by default, but deleted if the name of the declared subpattern entity is mentioned within a delete statement.
In replace mode, they are deleted by default, but kept if the name of the declared subpattern entity is mentioned (using occurence, same as with nodes or edges).

\begin{example}
  \begin{grgen}
rule R {
  m1:Method; m2:Method;
  tp1:TwoParameters(m1);
  tp2:TwoParameters(m2);

  replace {
    tp1; // is kept
    // tp2 not included here - will be deleted
    // tp1(); or tp2(); -- would apply dependent replacement
    m1; m2;
  }
}
  \end{grgen}
\end{example}

\begin{note}
You may even give a SubpatternEntityDeclaration within a rewrite part which causes the subpattern to be created; 
but this employment has several issues which can only be overcome by introducing explicit creation-only subpatterns
-- so you better only use it if you think it should obviously work 
and you are willing to adapt your code in a later version. 

  \begin{grgen}
pattern ForCreationOnly(mp:Method)
{
  // some complex pattern you want to instantiate several times 
  // connecting it to the mp handed in
}
rule createSubpattern
{
  m:Method;
  
  modify {
    :ForCreationOnly(m); // instantiate pattern ForCreationOnly
  }
}
  \end{grgen}
\end{note}


%%%%%%%%%%%%%%%%%%%%%%%%%%%%%%%%%%%%%%%%%%%%%%%%%%%%%%%%%%%%%%%%%%%%%%%%%%%%%%%%%%%%%%%%%%%%%%%%
\section{Regular Expression Syntax}\indexmainsee{EBNF}{regular expression syntax}\indexmain{regular expression syntax}

In addition to the already introduced syntax for the nested patterns with the keywords 
\texttt{negative}, \texttt{independent}, \texttt{alternative}, \texttt{iterated}, \texttt{multiple} and \texttt{optional},
there is a more lightweight syntax resembling regular expressions; 
using it together with the subpatterns gives graph rewrite specifications which look like EBNF-grammars with embedded actions. 
Exceeding the more verbose syntax they offer constructs for matching the pattern a bounded number of times (same notation as the one for the bounded iteration in the xgrs).
Table \ref{keywordregexpsyntax} lists the corresponding (/equivalent) language constructs; 
Example \ref{introexampleregexp} is a version of the introductionary example \ref{introexample} modified to use the new syntax.

\begin{table}[htbp]
  \centering
  \begin{tabularx}{\linewidth}{|l|X|} \hline
    \texttt{ iterated \{ P \} } & \texttt{ (P)* } \\
    \texttt{ multiple \{ P \} } & \texttt{ (P)+ } \\
    \texttt{ optional \{ P \} } & \texttt{ (P)? } \\ 
	\texttt{ alternative \{ l1 \{ P1 \} .. lk \{ Pk \} \} } & \texttt{ (P1|..|Pk) } \\
    \texttt{ negative \{ P \} } & \texttt{ $\sim$(P) } \\
    \texttt{ independent \{ P \} } & \texttt{ \&(P)} \\ \hline 
    \texttt{ modify \{ R \} } & \texttt{  \{+ R \} } \\
    \texttt{ replace \{ R \} } & \texttt{ \{- R \} } \\ \hline 
    \texttt{ - } & \texttt{ (P)[k] / (P)[k:l] / (P)[k:*] } \\ \hline 
	\end{tabularx}
  \caption{Map of nested patterns in keyword syntax to regular expression syntax}
  \label{keywordregexpsyntax}
\end{table}


  \begin{example}
    \begin{grgen}
test method
{
  m:Method <-- n:Name; // signature of method consisting of name
  ( m <-- :Variable; )* // and 0-n parameters
  
  :AssignmentList(m); // body consisting of a list of assignment statements
}

pattern AssignmentList(prev:Node)
{
  ( // nothing or a linked assignment and again a list
    prev --> a:Assign; // assignment node 
    a -:target-> v:Variable; // which has a variable as target 
    :Expression(a);  // and an expression which defines the left hand side 
    :AssignmentList(a); // next one, plz
  )?
}

pattern Expression(root:Expr)
{
  ( // expression may be a binary expression of an operator and two expresions
      root <-- expr1:Expr;
	  :Expression(expr1);
      root <-- expr2:Expr;
	  :Expression(expr2);
      root <-- :Operator;
  | // or a unary expression which is a variable (reading it)
      root <-- v:Variable;
  )
}
    \end{grgen}
  \end{example}\label{introexampleregexp}


%%%%%%%%%%%%%%%%%%%%%%%%%%%%%%%%%%%%%%%%%%%%%%%%%%%%%%%%%%%%%%%%%%%%%%%%%%%%%%%%%%%%%%%%%%%%%%%%
\section{Merge and Split, Node Replacement Grammars, and Subgraph copying}\label{sub:copymergesplit}

In this section we'll have a look at several common graph rewriting tasks and how to accomplish them in \GrG.
They could be offered directly by some dedicated operators,
but these would need so much customization to be useful in the different situations one needs them,
that we decided against dedicated operators; 
instead it is on you to program the version you need yourself by combining language constructs and rules.

There are two-and-a-half means available in the \GrG-rule language to build combined, complex rules:
\begin{description}
	\item[nested and subpatterns]
which allow to match and rewrite complex patterns built in a structured way piece by piece.
With different pieces connected together, pieces to decide in between, and pieces which appear repeatedly. 
	\item[embedded graph rewrite sequences]
for deferred rule execution to do work later on you can't do while executing the rule proper (e.g. because an element was already matched and is now locked due to the isomorphy constraint); or for splitting work into several parts, reusing already available functionality.
	\item[storages and storagemaps]
to store elements collected in one run and reuse them as input in another run, i.e. using multi-value variables to break up the transformation into passes. (This is the half point, as they are not a means to build a complex rule, but a means to communicate between rules building complex transformations.)
\end{description}

\noindent In the following we'll employ them to merge and split nodes, emulate node replacement grammars, and copy substructures.

\begin{note}
You should use the means to build complex transformations in the order given, first try to solve the problem with nested and subpatterns, if they don't get the job done (often because of the isomorphy constraint locking of already matched elements) or feel awkward then embedded graph rewrite sequences, and if these are still inadequate then storages and storagemaps; i.e. from declarative-local to imperative-global. This helps in keeping the code readable and easily adaptable. And don't forget the last resort if you must solve a task so complex that rules, control and storages are not sufficient: adding helper nodes, edges, or attributes to the graph model and the graph itself (with the visited flags being a light version of this tactic). Note: storages and storagemaps might be useful when you are experiencing performance problems, replacing a search in the graph by a lookup in a dictionary. But beware of elements already deleted from the graph still hanging out in your storage because you forgot to remove them.
\end{note}

\subsubsection*{Merge and split nodes}\indexmain{merge node}\indexmain{split node}
Merging a node \texttt{m} into a node \texttt{n} means transferring all edges from node \texttt{m} to node \texttt{n}, then deleting node \texttt{m}. 
Splitting a node \texttt{m} off from a node \texttt{n} means creating a node \texttt{m} and transferring some edges from node \texttt{n} to \texttt{m}.

In both cases there are a lot of different ways how to handle the operation exactly:
Maybe only incoming or only outgoing edges, or only edges of a certain type \texttt{T} or only edges not of type \texttt{T}; maybe the node \texttt{n} is to be retyped, maybe the edges are to be retyped. 
But common is the transferring of edges; this can be handled succintly by an \texttt{iterated} statement and the \texttt{copy} operator.
In case the node opposite to an edge may be incident to several such edges, one must use an \texttt{exec} instead, as every iteration locks the matched entities, so they can't get matched twice. Not needing the opposite node one could simply leave it unmentioned in the pattern, only referencing node \texttt{n} or \texttt{m} and the edge, but unfortunately we need the opposite node so we can connect the edge copy to it.

Now we'll have a look at an example for node merging: T1-T2 analysis from compiler construction is used to find out whether a control flow graph of a subroutine is reducible, i.e. all loops are natural loops. All loops being natural loops is a very useful property for many analyses and optimizations. The analysis is split into two steps, T1 removes reflexive edges, T2 merges a control flow successor into its predecessor iff there is only one predecessor available. These two steps are iterated until the entire graph is collapsed into one node which means the control flow is reducible, or execution gets stuck before, in which case the control flow graph is irreducible.
The analysis is defined on simple graphs, i.e. if two control flow edges between two basic block nodes appear because of merging they are seen as one, i.e. they are automatically fused into one. As \GrG~is built on multigraphs we have to explicitely do the edge fusion in a further step T3.

First let us have a look at T1 and T3, which are rather boring ... ehm, straight forward:

  \begin{example}
    \begin{grgen}
rule T1
{
  n:BB -:cf-> n;
  
  replace {
    n; // delete relexive edges
  }
}

rule T3
{
  pred:BB -first:cf-> succ:BB;
  pred    -other:cf-> succ;
  
  modify { // kill multiedges
    delete(other);
  }
}
    \end{grgen}
  \end{example}

The interesting part is T2, this is the first version using an iterated statement:

  \begin{example}
    \begin{grgen}
rule T2
{
  pred:BB -e:cf-> succ:BB;
  negative {
    -e->;
    -:cf-> succ; // if succ has only one predecessor
  }
  iterated {
    succ -ee:cf-> n:BB;
    
    modify { // then merge succ into this predecessor
      pred -:copy<ee>-> n; // copying the succ edges to pred
    }
  }
  
  modify { // then merge succ into this predecessor
    delete(succ);
  }
}
    \end{grgen}
  \end{example}

In case a control flow graph would be a multi-graph, with several control flow edges between two nodes, one would have to use an \texttt{exec} with an all-bracketed rule instead of the \texttt{iterated}, to be able to match a multi-\texttt{cf}-edge target of \texttt{succ} multiple times (which is prevented in the \texttt{iterated} version by the isomorphy constraint locking the target after the first match).

This is the second version using exec instead, capable of handling multi edges:

  \begin{example}
    \begin{grgen}
rule T2exec
{
  pred:BB -e:cf-> succ:BB;
  negative {
    -e->;
    -:cf-> succ; // if succ has only one predecessor
  }
  
  modify { // then merge succ into this predecessor
    exec([copyToPred(pred, succ)] ;> delSucc(succ));
  }
}

rule copyToPred(pred:BB, succ:BB)
{
  succ -e:cf-> n:BB;
  
  modify {
    pred -:copy<e>-> n;
  }
}

rule delSucc(succ:BB)
{
  modify {
    delete(succ);
  }
}
    \end{grgen}
  \end{example}

Natural loops are so advantageous that one transforms irreducible graphs (which only occur by using wild gotos) into reducible ones, instead of bothering with them in the analyses and optimizations.
An irreducible graph can be made reducible by node splitting, which amounts to code duplication (in the program behind the control flow graph).
In a stuck situation after T1-T2 analysis, a \texttt{BB} node with multiple control flow predecessors is split into as many nodes as there are control flow predecessors, every one having the same control flow successors as the original node.
(Choosing the \texttt{cf} edges and \texttt{BB} nodes which yield the smallest amount of code duplication is another problem which we happily ignore here.)

  \begin{example}
We do the splitting by keeping the indeterministically chosen first cf edge, splitting off only further cf edges, replicating their common target.

    \begin{grgen}
rule split(succ:BB)
{
  pred:BB -first:cf-> succ;
  multiple {
    otherpred:BB -other:cf-> succ;
    
    modify {
      otherpred -newe:cf-> newsucc:copy<succ>;
      delete(other);
      exec(copyCfSuccFromTo(succ, newsucc));
    }
  }
  
  modify {
  }
}

rule copyCfSuccFromTo(pred:BB, newpred:BB)
{
  iterated {
    pred -e:cf-> succ:BB;
    
    modify {
      newpred -:copy<e>-> succ;
    }
  }
  
  modify {
  }
}
    \end{grgen}
  \end{example}

The examples given can be found in the \texttt{tests/mergeSplit/} directory including the control scripts and test graphs; you may add \texttt{debug} prefixes to the \texttt{xgrs} statements in the graph rewrite script files and call GrShell with e.g. \texttt{mergeSplit/split.grs} as argument from the \texttt{tests} directory to watch execution.

\subsubsection*{Node Replacement Grammars}\indexmain{node replacement grammar}\indexmainsee{edNCE}{node replacement grammar}
With node replacement grammars we mean edNCE grammars \cite{NodeReplacement}, which stands for edge label directed node controlled embedding. In this context free graph grammar formalism, every rule describes how a node with a nonterminal type is replaced by a subgraph containing terminal and nonterminal nodes and terminal edges. The nodes in the instantiated graph get connected to the nodes that were adjacent to the initial nonterminal node, by connection instructions which tell which edges of what direction and what type are to be created for which original edges of what direction and what type, going to a node of what type. 

This kind of grammars can be encoded in \GrG~by rules with a left hand side consisting of a node with a type denoting a nonterminal and iterateds matching the edges and opposite nodes it is connected to of interest; "of interest" amounts to the type and direction of the edges and the type of the opposite node. The right hand side deletes the original node (thus implicitely the incident edges), creates the replacement subgraph, and tells in the rewrite part of the iterateds what new edges of what directedness and type are to be created, from the newly created nodes to the nodes adjacent to the original node. (Multiple edges between two nodes are not allowed in the node replacement formalism, in case you want to handle them you've to use \texttt{exec} as shown in the merge/split example above.) 

The following example directly follows this encoding:

  \begin{example}
This is an example rule replacing a nonterminal node \texttt{n:NT} by a 3-clique.
For the outgoing \texttt{E1} edges of the original node, the new node \texttt{x} receives incoming \texttt{E2} edges.
And for incoming \texttt{E2} edges of the original node, the new nodes \texttt{y} and \texttt{z} receive edges of the same type, \texttt{y} with reversed direction and \texttt{z} of the exact dynamic subtype bearing the same values as the original edges.
    \begin{grgen}
rule example
{
  n:NT;

  iterated {
    n -:E1-> m:T;

    modify {
      x <-:E2- m;
    }
  }

  iterated {
    n <-e2:E2- m:T;   	

    modify {
      y -:E2-> m;
      z <-:copy<e2>- m;      
    }
  }
  
  modify {
    delete(n);
    x:T -- y:T -- z:T -- x; 
  }
}
    \end{grgen}
  \end{example}

As another example for node replacement grammars we encode the two rules needed for the generation of the completely connected graphs (cliques) in two \GrG~rules. The first replaces the nonterminal node by a new nonterminal node linked to a new terminal node, connecting both new nodes to all the nodes the original nonterminal node was adjacent to. The second replaces the nonterminal node by a terminal node, connecting the new terminal node to all the nodes the original nonterminal node was adjacent to. This "we want to preserve the original edges" can be handled more succintly and efficiently by retyping which we gladly use instead of the iteration.

  \begin{example}
    \begin{grgen}
rule cliqueStep
{
  nt:NT;
  
  iterated {
    nt -- neighbour:T;

    modify {
      t -- neighbour;
      nnt -- neighbour;
    }
  }
  
  modify {
    delete(nt);
    t:T -- nnt:NT;
  }
}

rule cliqueTerminal
{
  nt:NT;
  
  modify {
    :T<nt>;
  }
}
    \end{grgen}
  \end{example}

The examples can be found in the \texttt{tests/nodeReplacementGrammar} directory.

\subsubsection*{Copy structures}\label{subsub:copystructure}\indexmain{subgraph copying}\indexmainsee{copy structure}{subgraph copying}
Structures are copied in two passes, the first copying and collecting all nodes of interest, the second copying all edges of interest in between the nodes.

The first pass consists of covering the nodes of the structure one wants to copy with iterated subpatterns,
i.e. subpatterns which match from a root node on with iterateds along the incident edges into breadth,
employing a subpattern again on the node opposite to the root node to match into depth.
In the example we match the entire subgraph from a root node on, if one wants to copy a more constrained subgraph one can simple constrain the types, directions, and structures in the iterated subpattern covering the nodes.
The nodes are copied with the \texttt{copy} operators and a storagemap is filled, storing for every node copied its copy.

The second pass is started after the structure matching ended by executing the deferred execs which were issued for every node handled. 
Each \texttt{exec} copies all outgoing edges (one could process all incoming edges instead) of a node:
for each edge leaving the original node towards another original node a copy is created in between the copy of the original node and the copy of the other node.
The copies are looked up with the original nodes from the storage map (which fails for target nodes outside of the subgraph of interest).
Here too one could constrain the subgraph copied by filtering certain edges.
In case of undirected edges one would have to prevent that edges get copied twice (once for every incident node). This would require a visited flag for marking the already copied edges or a storage receiving them, queried in the edge copying pattern and set/filled in the edge copying rewrite part.

  \begin{example}
The example shows very generally how a subgraph reachable from a root node by incident edges can get copied, collecting and copying the nodes along a spanning tree from the root node on, then copying the edges in between the nodes in a second run afterwards. The edges get connected to the correct node copies via a mapping from the old to the new nodes remembered in a storage-map.
    \begin{grgen}
pattern CopySubgraph(root:Node, ref oldToNew:map<Node, Node>)
{
  iterated { // match spanning tree of graph from root on
    root <--> ch:Node;   	
    cs:CopySubgraph(ch, oldToNew);
    
    modify {
      cs();
    }
  }
  
  modify {
    newroot:copy<root>; // copy nodes
    eval { oldToNew.add(root, newroot); }
    exec( [CopyOutgoingEdge(root, oldToNew)] ); // deferred copy edges
  }
}

rule CopyOutgoingEdge(n:Node, ref oldToNew:map<Node, Node>)
{
  n -e:Edge-> m:Node;
  hom(n,m); // reflexive edges
  nn:Node<oldToNew[n]>; nm:Node<oldToNew[m]>;
  hom(nn,nm); // reflexive edges
    
  modify {
    nn -ee:copy<e>-> nm;
  }
}
    \end{grgen}
  \end{example}

The example can be found in the \texttt{tests/copyStructure} directory.
Without storagemaps one would have to pollute the graph model with helper edges linking the original to the copied nodes.

