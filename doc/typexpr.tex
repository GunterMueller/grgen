\chapter{Basic Types and Attribute Evaluation Expressions}
\label{cha:typeexpr}
%In the following sections \emph{Ident} refers to an identifier of the graph model language (see Section~\ref{modelbb}) or the rule set language (see Section~\ref{rulebb}).
%\emph{TypeIdent} is an identifier of a node type or an edge type, \emph{NodeOrEdge} is an identifier of a node or an edge.


%%%%%%%%%%%%%%%%%%%%%%%%%%%%%%%%%%%%%%%%%%%%%%%%%%%%%%%%%%%%%%%%%%%%%%%%%%%%%%%%%%%%%%%%%%%%%%%%
\section{Built-In Types}
\label{sec:builtintypes}
Besides user-defined node types, edge types, and enumeration types (as introduced in Chapter~\ref{chapmodellang}), \GrG\ supports the built-in \indexed{primitive types}\indexmainsee{built-in types}{primitive types} in Table~\ref{builtintypes} (and built-in generic types, cf. \ref{cha:container}).
The exact type format is \indexed{backend} specific.
The \indexed{LGSPBackend} maps the \GrG\ primitive types to the corresponding C\# primitive types.

\begin{table}[htbp]
\begin{tabularx}{\linewidth}{|l|X|}
	\hline
	\texttt{\indexed{boolean}} & Covers the values \texttt{true} and \texttt{false} \\
  \texttt{\indexed{byte, short, int, long}} & A signed integer, with 8 bits, with 16 bits, with 32 bits, with 64 bits \\
	\texttt{\indexed{float}}, \texttt{\indexed{double}} & A floating-point number, with single precision, with double precision\\
	\texttt{\indexed{string}} & A character sequence of arbitrary length\\
	\texttt{\indexed{object}} & Contains a .NET object\\
	\hline
\end{tabularx}
\caption{\GrG\ built-in primitive types}
\label{builtintypes}
\end{table}

\begin{table}[htbp]
  \centering
  \begin{tabular}[c]{|c|ccccccc|} \hline
    \backslashbox{to}{from} & \texttt{enum} & \texttt{boolean} & \texttt{int} & \texttt{double} & \texttt{string} & \texttt{object} & \\ \hline
    \texttt{enum} & $=$/--- & & & & & &\\
    \texttt{boolean} & & $=$ & & & & &\\
    \texttt{int} & implicit & & $=$ & \texttt{(int)} & & & \\
    \texttt{double} &  implicit & & implicit & $=$ & & & \\
    \texttt{string} & implicit & implicit & implicit &  implicit & $=$ & implicit & \\
    \texttt{object} & \texttt{(object)} & \texttt{(object)} & \texttt{(object)} & \texttt{(object)} & \texttt{(object)} & $=$ &\\\hline
  \end{tabular}
  \caption{\GrG\ type casts}
  \label{tabcasts}
\end{table}

Table~\ref{tabcasts} lists \GrG's implicit \indexed{type cast}s and the allowed explicit type casts.
Of course you are free to express an implicit type cast by an explicit type cast as well as ``cast'' a type to itself.
The \texttt{int} is the default integer type, in the table it stands for all the integer types.
The \texttt{double} is the default floating point type, in the table it stands for all the floating points types.
The integer types are implicitly casted upwards from smaller to larger types (\texttt{byte < short < int < long}), whereas a downcast requires an explicit cast.
The \texttt{byte} and \texttt{short} types are not used in computations, they are casted up to \texttt{int} (or \texttt{long} if required by the context.)
The floating point types are implicitly casted upwards (\texttt{float < double}), and require an explicit cast downwards, too.
As specified by the table, integer numbers are automatically casted to floating point numbers, and castable with an explicit cast vice versa.
According to the table neither implicit nor explicit casts from {\tt int} to any \indexed{enum type} are allowed.
This is because the range of an enum type is very sparse in general.
For the same reason implicit and explicit casts between enum types are also forbidden.
Thus, enum values can only be assigned to attributes having the same enum type.
A cast of an enum value to a string value will return the declared name of the enum value.
A cast of an object value to a string value will return ``null'' or it will call the \texttt{toString()} method of the .NET object.
Everything is implicitly casted to \texttt{string} to enable concise text output without the need for boilerplate casting.
Be careful with assignments of objects: \GrG\ does not know your .NET type hierarchy and therefore it cannot check two objects for type compatibility.
Objects of type object are not very useful for \GrG\ processing and the im/exporters can't handle them,
but they can be used on the API level.

\begin{example}
  \begin{itemize}
    \item Allowed:\\
	  \texttt{x.myfloat = x.myint; x.mydouble = (float) x.myint;\\ x.mystring = (string) x.mybool;}
    \item Forbidden:\\
      \texttt{x.myfloat = x.mydouble;} and \texttt{x.myint = (int) x.mybool;}\\
      \texttt{MyEnum1 = (MyEnum1Type) int;} and \texttt{MyEnum2 = (MyEnum2Type) MyEnum1;}
  where {\tt myenum1} and {\tt myenum2} are different enum types.

  \end{itemize}
\end{example}

\begin{note}
	Unlike an {\tt eval} part (which must not contain assignments to node or edge attributes) the declaration of an enum type can contain assignments of {\tt int} values to \indexed{enum item}s (see Section~\ref{typedecl}).
	The reason is, that the range of an enum type is just defined in that context.
\end{note}


%%%%%%%%%%%%%%%%%%%%%%%%%%%%%%%%%%%%%%%%%%%%%%%%%%%%%%%%%%%%%%%%%%%%%%%%%%%%%%%%%%%%%%%%%%%%%%%%
\section{Expressions}\indexmain{expression}\label{sub:expr}

\GrG~supports numerous operations on the entities of the types introduced above, which are organized into left associative expressions.
In the following they will be explained with their semantics and relative priorities one type after another in the order of the rail diagram below.

\begin{rail}
  Expression: BoolExpr | RelationalExpr | IntExpr | FloatExpr | StringExpr | TypeExpr | PrimaryExpr;
\end{rail}\ixnterm{Expression}


%%%%%%%%%%%%%%%%%%%%%%%%%%%%%%%%%%%%%%%%%%%%%%%%%%%%%%%%%%%%%%%%%%%%%%%%%%%%%%%%%%%%%%%%%%%%%%%%
\section{Boolean Expressions}

The boolean expressions combine boolean values with logical operations.
They bind weaker than the relational expressions which bind weaker than the other expressions.

\begin{rail}
  BoolExpr: ((() | '!') PrimaryExpr) | (BoolExpr '?' BoolExpr ':' BoolExpr) | (BoolExpr BinBoolOperator BoolExpr) | RelationalExpr;
\end{rail}\ixnterm{BoolExpr}
The unary \texttt{!}\ operator negates a Boolean.
The binary \emph{BinBoolOperator} is one of the operators in Table~\ref{tabboolops}.
\begin{table}[htbp]
  \centering
  %\begin{tabularx}{0.45\linewidth}{|ll|} \hline
  \begin{tabular}[c]{|lp{0.6\linewidth}|} \hline
    \begin{tabular}[c]{l} \texttt{\^} \end{tabular} & \begin{tabular}[c]{l} Logical XOR. True, iff either the first or the second \\ Boolean expression is true. \end{tabular} \\ \hline
    \begin{tabular}[c]{l} \texttt{\&\&} \\ \texttt{||} \end{tabular} & \begin{tabular}[c]{l} Logical AND and OR. Lazy evaluation. \end{tabular}\\ \hline
    \begin{tabular}[c]{l} \texttt{\&} \\ \texttt{|} \end{tabular} & \begin{tabular}[c]{l} Logical AND and OR. Strict evaluation. \end{tabular}\\ \hline
  \end{tabular}
  \caption{Binary Boolean operators, in ascending order of precedence}\indexmain{order of precedence}\indexmainsee{precedence}{order of precedence}
  \label{tabboolops}
\end{table}
The ternary \texttt{?}\ operator is a simple if-then-else: If the first \emph{BoolExpr} is evaluated to \texttt{true}, the operator returns the second \emph{BoolExpr}, otherwise it returns the third \emph{BoolExpr}.


%%%%%%%%%%%%%%%%%%%%%%%%%%%%%%%%%%%%%%%%%%%%%%%%%%%%%%%%%%%%%%%%%%%%%%%%%%%%%%%%%%%%%%%%%%%%%%%%
\section{Relational Expressions}\label{sec:relational}

The relational expressions compare entities of different kinds, mapping them to the type boolean.
They are a middle between the boolean type of decisions and the other types.
The bind stronger than the boolean expressions but weaker than all the other non-boolean expressions.

\begin{rail}
 RelationalExpr: (Expression CompareOperator Expression)
\end{rail}\ixnterm{RelationalExpr}

The \emph{CompareOperator} is one of the following operators:
\[ \texttt{<} \;\;\;\;\; \texttt{<=} \;\;\;\;\; \texttt{==} \;\;\;\;\; \texttt{!=} \;\;\;\;\; \texttt{>=} \;\;\;\;\; \texttt{>} \;\;\;\;\; \texttt{\textasciitilde\textasciitilde} \]
Their semantics are type dependent.

For arithmetic expressions on \texttt{int} and \texttt{float} or \texttt{double} types
the semantics is given by Table~\ref{compandarithmetic} (by implicit casting they can also by used with all enum types).

\begin{table}[htbp]
  \centering
  \begin{tabularx}{\linewidth}{|l|X|} \hline
    \texttt{A == B} & True, iff $A$ is the same number as $B$. \\
    \texttt{A != B} & True, iff $A$ is a different number than $B$. \\
    \texttt{A <\ \ B} & True, iff $A$ is smaller than and not equal $B$. \\
    \texttt{A >\ \ B} & True, iff $A$ is greater than and not equal $B$. \\
    \texttt{A <= B} & True, iff $A$ is smaller than (or equal) $B$. \\
    \texttt{A >= B} & True, iff $A$ is greater than (or equal) $B$. \\ \hline
  \end{tabularx}
  \caption{Compare operators on arithmetic expressions}
  \label{compandarithmetic}
\end{table}

For expressions on \texttt{string} types lexicographic order is used for comparisons,
the exact semantics is given by Table~\ref{compandstring}.

\begin{table}[htbp]
  \centering
  \begin{tabularx}{\linewidth}{|l|X|} \hline
    \texttt{A == B} & True, iff $A$ is the same string as $B$. \\
    \texttt{A != B} & True, iff $A$ is not the same string as $B$. \\
    \texttt{A <\ \ B} & True, iff the first character where $A$ and $B$ differ is smaller for $A$, or $A$ is a prefix of $B$. \\
    \texttt{A >\ \ B} & True, iff the first character where $A$ and $B$ differ is smaller for $B$, or $B$ is a prefix of $A$. \\
    \texttt{A <= B} & True, iff the first character where $A$ and $B$ differ is smaller for $A$, or $A$ is a prefix of $B$, or $A$ is the same as $B$. \\
    \texttt{A >= B} & True, iff the first character where $A$ and $B$ differ is smaller for $B$, or $B$ is a prefix of $A$, or $B$ is the same as $A$. \\ \hline
  \end{tabularx}
  \caption{Compare operators on string expressions}
  \label{compandstring}
\end{table}

\texttt{Boolean} types and \texttt{object} types support only the \texttt{==} and the \texttt{!=} operators;
on boolean values they denote equivalence and antivalence,
and on object types they tell whether the references are the same, thus the objects identical.

For \indexed{type expression}s the semantics of compare operators are given by table~\ref{compandtypes},
the rule to remember is: types grow larger with extension/refinement. An example is given in \ref{typeexpressions}.
\begin{table}[htbp]
  \centering
  \begin{tabularx}{\linewidth}{|l|X|} \hline
    \texttt{A == B} & True, iff $A$ and $B$ are identical. Different types in a type hierarchy are \emph{not} identical. \\
    \texttt{A != B} & True, iff $A$ and $B$ are not identical. \\
    \texttt{A <\ \ B} & True, iff $A$ is a supertype of $B$, but $A$ and $B$ are not identical. \\
    \texttt{A >\ \ B} & True, iff $A$ is a subtype of $B$, but $A$ and $B$ are not identical. \\
    \texttt{A <= B} & True, iff $A$ is a supertype of $B$ or $A$ and $B$ are identical. \\
    \texttt{A >= B} & True, iff $A$ is a subtype of $B$ or $A$ and $B$ are identical. \\ \hline
  \end{tabularx}
  \caption{Compare operators on type expressions}
  \label{compandtypes}
\end{table}
\begin{note}
  \texttt{A < B} corresponds to the direction of the arrow in an \indexed{UML class diagram}.
\end{note}
\begin{note}
  \texttt{Node} and \texttt{Edge} are the least specific, thus bottom elements $\bot$ of the type hierarchy,\\
  i.e. the following holds:
  \begin{itemize}
    \item $\forall n\in Types_{Node}: Node <= n$
    \item $\forall e\in Types_{Edge}: Edge <= e$
  \end{itemize}
\end{note}

%%%%%%%%%%%%%%%%%%%%%%%%%%%%%%%%%%%%%%%%%%%%%%%%%%%%%%%%%%%%%%%%%%%%%%%%%%%%%%%%%%%%%%%%%%%%%%%%
\section{Arithmetic and Bitwise Expressions}

The arithmetic and bitwise expressions combine integer and floating point values with the arithmetic operations usually available in programming languages
and integer values with bitwise logical operations (interpreting integer values as bit-vectors).

\begin{rail}
  IntExpr: ((() | '+' | '-' | tilde) PrimaryExpr) | (BoolExpr '?' IntExpr ':' IntExpr) | (IntExpr BinIntOperator IntExpr);
\end{rail}\ixnterm{IntExpr}
The $\sim$ operator is the bitwise complement.
That means every bit of an integer value will be flipped.
The \texttt{?}\ operator is a simple if-then-else: If the \emph{BoolExpr} is evaluated to \texttt{true}, the operator returns the first \emph{IntExpr}, otherwise it returns the second \emph{IntExpr}.
The \emph{BinIntOperator} is one of the operators in Table~\ref{tabbinops}.
\begin{table}[htbp]
  \centering
  %\begin{tabularx}{0.45\linewidth}{|ll|} \hline
  \begin{tabular}[c]{|lp{0.6\linewidth}|} \hline
    \begin{tabular}[c]{l} \texttt{\^} \\ \texttt{\&} \\ \texttt{|} \end{tabular} & \begin{tabular}[c]{l} Bitwise XOR, AND and OR \end{tabular} \\ \hline
    \begin{tabular}[c]{l} \texttt{\mbox{<}\mbox{<}} \\ \texttt{\mbox{>}\mbox{>}} \\ \texttt{\mbox{>}\mbox{>}\mbox{>}} \end{tabular} & \begin{tabular}[c]{l} Bitwise shift left, bitwise shift right and \\ bitwise shift right prepending zero bits (unsigned mode)\end{tabular}\\ \hline
    \begin{tabular}[c]{l} \texttt{+} \\ \texttt{-} \end{tabular} & \begin{tabular}[c]{l} Addition and subtraction \end{tabular}\\ \hline
    \begin{tabular}[c]{l} \texttt{*} \\ \texttt{/} \\ \texttt{\%} \end{tabular} & \begin{tabular}[c]{l}Multiplication, integer division, and modulo \end{tabular} \\ \hline
  \end{tabular}
  \caption{Binary integer operators, in ascending order of precedence}\indexmain{order of precedence}
  \label{tabbinops}
\end{table}

If one operand is \texttt{long} the operation is carried out with 64 Bits, otherwise the operation is carried out with 32 Bits, i.e. \texttt{int}-sized --- even if all the operands are of type \texttt{byte} or \texttt{short}.

\begin{rail}
  FloatExpr: ((() | '+' | '-') PrimaryExpr) | (BoolExpr '?' FloatExpr ':' FloatExpr) | (FloatExpr BinFloatOperator FloatExpr);
\end{rail}\ixnterm{FloatExpr}
The \texttt{?}\ operator is a simple if-then-else: If the \emph{BoolExpr} is evaluated to \texttt{true}, the operator returns the first \emph{FloatExpr}, otherwise it returns the second \emph{FloatExpr}.
The \emph{BinFloatOperator} is one of the operators in Table~\ref{tabfloatbinops}.
\begin{table}[htbp]
  \centering
  %\begin{tabularx}{0.45\linewidth}{|ll|} \hline
  \begin{tabular}[c]{|ll|} \hline
    \begin{tabular}[c]{l} \texttt{+} \\ \texttt{-} \end{tabular} & \begin{tabular}[c]{l} Addition and subtraction \end{tabular}\\ \hline
    \begin{tabular}[c]{l} \texttt{*} \\ \texttt{/} \\ \texttt{\%} \end{tabular} & \begin{tabular}[c]{l}Multiplication, division and modulo \end{tabular} \\ \hline
  \end{tabular}
  \caption{Binary float operators, in ascending order of precedence}\indexmain{order of precedence}
  \label{tabfloatbinops}
\end{table}
\begin{note}
The \texttt{\%} operator on float values works analogous to the integer modulo operator.
For instance \texttt{4.5 \% 2.3 == 2.2}.
\end{note}

If one operand is \texttt{double} the operation is carried out with 64 Bits, otherwise the operation is carried out with 32 Bits, i.e. \texttt{float}-sized.


%%%%%%%%%%%%%%%%%%%%%%%%%%%%%%%%%%%%%%%%%%%%%%%%%%%%%%%%%%%%%%%%%%%%%%%%%%%%%%%%%%%%%%%%%%%%%%%%
\section{String Expressions}

String expressions combine string values by string operations, with integer numbers used as helpers to denote positions in the strings (and giving the result of length counting).

\begin{rail}
  StringExpr: PrimaryExpr (MethodSelector)? | StringExpr '+' StringExpr ;
  MethodSelector: '.' MethodName '(' (Variable*',') ')' ;
\end{rail}\ixnterm{StringExpr}\ixnterm{MethodSelector}
The operator \texttt{+} concatenates two strings.
There are several operations on strings available in method call notation (MethodSelector), these are

\begin{description}
\item[\texttt{.length()}] returns length of string, as \texttt{int}
\item[\texttt{.indexOf(strToSearchFor)}] returns first position where \texttt{strToSearchFor:string} appears at, as \texttt{int}, or -1 if not found
\item[\texttt{.indexOf(strToSearchFor, startIndex)}] returns first position where \texttt{strToSearchFor:string} appears at, when we start the search for it at string position \texttt{startIndex:int}, as \texttt{int}, or -1 if not found
\item[\texttt{.lastIndexOf(strToSearchFor)}] returns last position \texttt{strToSearchFor:string} appears at, as \texttt{int}, or -1 if not found
\item[\texttt{.substring(startIndex, length)}] returns substring of given \texttt{length:int} from \texttt{startIndex:int} on
\item[\texttt{.replace(startIndex, length, replaceStr)}] returns string with substring from \texttt{startIndex:int} on of given \texttt{length:int} replaced by \texttt{replaceStr:int}
\end{description}

\begin{example}
For \texttt{n.str == "foo bar foo"} the operations yield \\
\texttt{n.str.length()==11} \\
\texttt{n.str.indexOf("foo")==0} \\
\texttt{n.str.lastIndexOf("foo")==8} \\
\texttt{n.str.substring(4,3)=="bar"} \\
\texttt{n.str.replace(4,3,"foo")=="foo foo foo"} \\
\end{example}


%%%%%%%%%%%%%%%%%%%%%%%%%%%%%%%%%%%%%%%%%%%%%%%%%%%%%%%%%%%%%%%%%%%%%%%%%%%%%%%%%%%%%%%%%%%%%%%%
\section{Type Expressions}\indexmain{type expression}
\label{typeexpressions}

\begin{rail}
  TypeExpr: TypeIdent | 'typeof' '(' NodeOrEdge ')' ;
\end{rail}\ixkeyw{typeof}\ixnterm{TypeExpr}
A type expression identifies a type (and---in terms of matching---also its subtypes).
A type expression is either a type identifier itself or the type of a graph element.
The type expression \texttt{typeof(x)} stands for the type of the host graph element \texttt{x} is actually bound to.

\begin{example}
\begin{tabularx}{\linewidth}{cX}
  \begin{tikzpicture}[baseline=(T.base)] \tt
    \begin{scope}[minimum size=0.5cm]
      \tikzstyle{every node}=[draw]
      \node (T)     at (1   ,4) {\texttt{T}};
      \node (T1)     at (1   ,3) {\texttt{T1}};
      \node (T2)     at (0   ,2) {\texttt{T2}};
      \node (T4)     at (0   ,1) {\texttt{T4}};
      \node (T3)     at (2   ,2) {\texttt{T3}};
    \end{scope}
    \draw[thick,-open triangle 45]  (T1) -> (T)  ;
    \draw[thick,-open triangle 45]  (T2) -> (T1)  ;
    \draw[thick,-open triangle 45]  (T3) -> (T1)  ;
    \draw[thick,-open triangle 45]  (T4) -> (T2)  ;
  \end{tikzpicture} &
  \parbox{\linewidth}{The expression \texttt{typeof(x)<=T2} applied to the type hierarchy on the left side yields \texttt{true} if \texttt{x} is a graph element of type \texttt{T} or \texttt{T1} or \texttt{T2}.
                      The expression \texttt{typeof(x)>T2} only yields \texttt{true} for \texttt{x} being a graph element of type \texttt{T4}. The expression \texttt{T1<T3} always yields \texttt{true}.}
\end{tabularx}
\end{example}


%%%%%%%%%%%%%%%%%%%%%%%%%%%%%%%%%%%%%%%%%%%%%%%%%%%%%%%%%%%%%%%%%%%%%%%%%%%%%%%%%%%%%%%%%%%%%%%%
\section{Primary Expressions}\label{sec:primexpr}

After we've seen the all the ways to combine expressions, finally we'll have a look at the atoms the expressions are built of.

\begin{rail}
  PrimaryExpr: CastExpr
  		| MemberAccess
  		| FunctionCall
  		| VarIdent
  		| '::' GlobalVarIdent
  		| 'nameof' '(' NodeOrEdge? ')'
  		| 'uniqueof' '(' NodeOrEdge? ')'
  		| 'random' '(' ( IntExpr )? ')'
  		| '(' Expression ')';
\end{rail}\ixnterm{PrimaryExpr}\ixkeyw{nameof}\indexmainsee{variable}{expression variable}\indexmain{expression variable}\indexmain{variable}

The \texttt{nameof} query returns the name (persistent name, see example \ref{persistentex}) of the given graph element as \texttt{string}
(graphs elements of pure \texttt{LGSPGraph}s bear no name, then the query fails, but normally you are using \texttt{LGSPNamedGraph}s).
Besides nodes or edges, subgraphs may be given, then the name of the subgraph is returned.
If no argument is given, the name of the graph (the current host graph) is returned.

The \texttt{uniqueof} query returns the unique id (see \ref{sec:uniqueness}) of the given graph element as \texttt{int}
(you only receive a unique id in case uniqueness was declared in the model).
Besides nodes or edges, subgraphs may be given, then the unique id of the subgraph is returned.
If no argument is given, the unique if of the graph (the current host graph) is returned.

The \texttt{random} function returns a double random value in between 0.0 and 1.0 if called without an argument,
or, if an integer argument value is given, an integer random value in between 0 and the value given, excluding the value itself.

\begin{rail}
  CastExpr: '(' PrimitiveType ')' PrimaryExpr;
  MemberAccess: (NodeOrEdge '.' Ident);
  FunctionCall: FunctionName '(' (() | Expr + ',') ')';
\end{rail}\ixnterm{CastExpr}\ixnterm{MemberAccess}\ixnterm{FunctionCall}

The cast expression returns the original value casted to the new prefixed type.
The member access \texttt{n.a} returns the value of the attribute \texttt{a} of the graph element \texttt{n}.
A function call employs an external (attribute evaluation) function (cf. \ref{sub:extfct}), or a (internal) user-defined function (see \ref{sub:functions}), or one of the following built-in functions contained in the built-in package \texttt{Math} or in the built-in package \texttt{Time} (for more on packages see \ref{sub:packageaction}):

\begin{description}
\item[\texttt{Math::min(.,.)}] returns the smaller of the two argument values, which must be of the same numeric type (i.e. both either \texttt{byte} or \texttt{short} or \texttt{int} or \texttt{long} or \texttt{float} or \texttt{double})
\item[\texttt{Math::max(.,.)}] returns the greater of the two argument values, which must be of the same numeric type (i.e. both either \texttt{byte} or \texttt{short} or \texttt{int} or \texttt{long} or \texttt{float} or \texttt{double})
\item[\texttt{Math::abs(.)}] returns the absolute value of the argument, which must be of numeric type (i.e. \texttt{byte} or \texttt{short} or \texttt{int} or \texttt{long} or \texttt{float} or \texttt{double})
\item[\texttt{Math::ceil(.)}] returns the ceiling of the argument, which must be of type \texttt{double}
\item[\texttt{Math::floor(.)}] returns the floor of the argument, which must be of type \texttt{double}
\item[\texttt{Math::round(.)}] returns the argument rounded, which must be of type \texttt{double}
\item[\texttt{Math::truncate(.)}] returns the argument truncated, which must be of type \texttt{double}
\item[\texttt{Math::pow(.,.)}] returns the first argument value to the power of the second value; both must be of type \texttt{double}
\item[\texttt{Math::pow(.)}] returns $e$ to the power of the argument value, which must be of type \texttt{double}
\item[\texttt{Math::log(.,.)}] returns the logarithm of the first argument value regarding the base given by the second value; both must be of type \texttt{double}
\item[\texttt{Math::log(.,.)}] returns the logarithm of the argument value that must be of type \texttt{double} regarding the base $e$
\item[\texttt{Math::sgn(.)}] returns the signum of the argument, which must be of type \texttt{double} ($-1$ if negative, $1$ if positive, $0$ if zero)
\item[\texttt{Math::sin(.)}] returns the sine of the argument, which must be of type \texttt{double}
\item[\texttt{Math::cos(.)}] returns the cosine of the argument, which must be of type \texttt{double}
\item[\texttt{Math::tan(.)}] returns the tangent of the argument, which must be of type \texttt{double}
\item[\texttt{Math::arcsin(.)}] returns the inverse of the sine for the argument, which must be of type \texttt{double}
\item[\texttt{Math::arccos(.)}] returns the inverse of the cosine for the argument, which must be of type \texttt{double}
\item[\texttt{Math::arctan(.)}] returns the inverse of the tangent for the argument, which must be of type \texttt{double}
\item[\texttt{Math::pi()}] returns the constant $\pi$, of type \texttt{double}
\item[\texttt{Math::e()}] returns the constant $e$, of type \texttt{double}
\end{description}

\begin{description}
\item[\texttt{Time::now(.)}] returns the current UTC time as \texttt{long}, given as windows file time (i.e. 100ns ticks since 1601-01-01)
\end{description}

\begin{rail}
  Literal: Constant | ContainerConstructor ;
  Constant: EnumLit | Number | HexNumber | FloatingNumber | QuotedText | BoolLit | NullLit;
\end{rail}\ixnterm{Constant}\ixnterm{Literal}\label{literaldef}

The Constants are:
\begin{description}
  \item[EnumLit] Is the value of an enum type, given in notation \texttt{EnumType '::' EnumValue}.
  \item[Number] Is a \texttt{byte} or \texttt{short} or \texttt{int} or \texttt{long} number in decimal notation without decimal point, postfixed by \texttt{y} or \texttt{Y} for \texttt{byte}, postfixed by \texttt{s} or \texttt{S} for \texttt{short}, postfixed by \texttt{l} or \texttt{L} for \texttt{long}, or of \texttt{int} type if not postfixed.
  \item[HexNumber] Is a \texttt{byte} or \texttt{short} or \texttt{int} or \texttt{long} number in hexadecimal notation starting with \texttt{0x}, the different types are distinguished by the suffix as for a decimal notation number.
  \item[FloatingNumber] Is a \texttt{float} or \texttt{double} number in decimal notation with decimal point, postfixed by \texttt{f} or \texttt{F} for \texttt{float}, maybe postfixed by \texttt{d} or \texttt{D} for \texttt{double} .
  \item[QuotedText] Is a string constant. It consists of a sequence of characters, enclosed by double quotes.
  \item[BoolLit] Is a constant of boolean type, i.e. one of the literals \texttt{true} or \texttt{false}.
  \item[NullLit] Is the one constant of object type, the literal \texttt{null}.
\end{description}

\begin{example}
Some examples of literals:
\begin{grgen}
Apple::ToffeeApple // an enum literal
42y // an integer number in decimal notation of byte type
42s // an integer number in decimal notation of short type
42 // an integer number in decimal notation of int type
42L // an integer number in decimal notation of long type
0x7eadbeef // an integer number in hexadecimal notation of int type
0xdeadbeefL // an integer number in hexadecimal notation of long type
3.14159 // a double number
3.14159f // a float number
"ve rule and 0wn ze vorld" // a text literal
true // a bool literal
null // the object literal
\end{grgen}
\end{example}

\pagebreak

%%%%%%%%%%%%%%%%%%%%%%%%%%%%%%%%%%%%%%%%%%%%%%%%%%%%%%%%%%%%%%%%%%%%%%%%%%%%%%%%%%%%%%%%%%%%%%%%
\section{Operator Priorities}

The priorities of all available operators are shown in ascending order in the table below, the dots mark the positions of the operands, the commas separate the operators available on the respective priority level.

\begin{table}[htbp]
  \centering
  \begin{tabular}[c]{|ll|} \hline
    \begin{tabular}[c]{l} 01 \end{tabular} & \begin{tabular}[c]{l} \verb#. ? . : .# \end{tabular}\\ \hline
    \begin{tabular}[c]{l} 02 \end{tabular} & \begin{tabular}[c]{l} \verb#. || .# \end{tabular}\\ \hline
    \begin{tabular}[c]{l} 03 \end{tabular} & \begin{tabular}[c]{l} \verb#. && .# \end{tabular}\\ \hline
    \begin{tabular}[c]{l} 04 \end{tabular} & \begin{tabular}[c]{l} \verb#. | .# \end{tabular}\\ \hline
    \begin{tabular}[c]{l} 05 \end{tabular} & \begin{tabular}[c]{l} \verb#. ^ .# \end{tabular}\\ \hline
    \begin{tabular}[c]{l} 06 \end{tabular} & \begin{tabular}[c]{l} \verb#. & .# \end{tabular}\\ \hline
    \begin{tabular}[c]{l} 07 \end{tabular} & \begin{tabular}[c]{l} \verb#. \ .# \end{tabular}\\ \hline
    \begin{tabular}[c]{l} 08 \end{tabular} & \begin{tabular}[c]{l} \verb#. ==,!= .# \end{tabular}\\ \hline
    \begin{tabular}[c]{l} 09 \end{tabular} & \begin{tabular}[c]{l} \verb#. <,<=,>,>=,in .# \end{tabular}\\ \hline
    \begin{tabular}[c]{l} 10 \end{tabular} & \begin{tabular}[c]{l} \verb#. <<,>>,>>> .# \end{tabular}\\ \hline
    \begin{tabular}[c]{l} 11 \end{tabular} & \begin{tabular}[c]{l} \verb#. +,- .# \end{tabular}\\ \hline
    \begin{tabular}[c]{l} 12 \end{tabular} & \begin{tabular}[c]{l} \verb#. *,%,/ .# \end{tabular}\\ \hline
    \begin{tabular}[c]{l} 13 \end{tabular} & \begin{tabular}[c]{l} \verb#~,!,-,+ .# \end{tabular}\\ \hline
  \end{tabular}
  \caption{All operators, in ascending order of precedence}\indexmain{order of precedence}
  \label{tabopprios}
\end{table}

